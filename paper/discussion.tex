In the following, we shortly discuss some of the raised issues and in how far they can be adressed in the future and possibly in the short-term.
\vsnote{Do we?}

The first challenge is that due to the way \Refinity{} prepares the environment and the top-level proof obligation for KeY,
in fact both sides in a refactoring share the same Java namespace.
This means for example that the \rname{Extract Method} refactoring no so much proves the original correct, but rather a version where the extracted method is already present.
Care must be taken to set this up correctly, and e.g.\ in the \rname{Extract Method} example make sure that the methods is not used on the \textit{Before}-side.
The same holds for refactorings that remove code.
\vsnote{Underwhelming.}

Adressing this would requiring choosing unique names in either schemata, since they go into a single KeY-proof, and hence would require some form of mapping classes/objects of distinct types (due to the nominal type system of Java) between both sides.
This issue is closely related to the challenge we discussed before when trying to relate unrelated yet semantically equivalent data types (see Section \ref{sec:relational}).
In general, we observe that currently \Refinity{} requires some scaffolding that makes the actual refactoring less obvious,
such as our use of assertions and casts.

A more general problem is capturing the most general instance of a refactoring.
Currently, \Refinity{} lacks placeholders for names, which means that e.g.\ in the \rname{Extract Local Variable} refactoring we would have to instantiate the \textit{Before}-schema for every concrete instance with the corresponding identifier names for variables.
Likewise, the related \rname{Hide Delegate} example uses concrete method- and class names.
During our development of this refactoring, we have found us revising the encoding and use of placeholders with their annotations repeatedly.
Conversely, due to the challenges in checking instantiation of schemata against concrete programs already pointed out by Steinhöfel \cite{steinhoefel-20},
\vsnote{Maybe add page number -- or is this also mentioned in the paper. Ole?}
one has to take care not to write too restrictive programs the rule out useful working instances

Another area for placeholders would be a generalisation of storage locations:
we foresee that there exist refactorings that may need to be specified twice, once with using local variables and another time using attributes in their schemata.
\vsnote{@EK: Can you confirm that I'm making sense here? It turns out that none of our refactorings is actually an example for that.}
\vsnote*{}{Eduard, are variables on the top-level of the method-context (and hence in Problem.java) equivalent to \textit{free program vars} in the sense that KeY always splits accesses into null/not-null when there is no concrete value? That was a bit unclear to us, since although in principle an AS could update them with a random value, we didn't see this explicitly in the proofs, or we missed it.}

\vsnote*{}{Not sure if the following should be here or in future work -- which we don't have as of yet?}
For the time being we are limited to checking schemata against each other.
In the future, when we move on to checking instantiations, 
we feel that often necessary pre-conditions on a refactoring can easily be discharged by simple syntactical or static analysis (e.g. ``code does never read attribute \lstinline|x| of objects of type \lstinline|C|'').
Yet unlike in other formal work where the program is encoded as part of the proof-term, we cannot implement such analyses within KeY, and can only informally document and require such side-conditions on refactorings.
Correspondingly, we will also not be able to use KeY to formulate and prove a lemma that such a static property entails the necessary consequences.
\vsnote*{}{Note to Eduard: what I mean here is that e.g. in Coq, if I would have the Java AST, I could implement \textbf{AND} prove as correct a static analysis. Can you add a sentence conjecturing on what would be possible if the code would be available as part of the proof-context? Or maybe it already is in some way?}


%%% Local Variables:
%%% mode: latex
%%% eval: (auto-fill-mode -1)
%%% TeX-master: "main"
%%% End:
