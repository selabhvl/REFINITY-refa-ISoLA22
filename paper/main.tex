\documentclass[runningheads]{llncs}


%%%%%%%%%%%%%%%%%
% USED PACKAGES %
%%%%%%%%%%%%%%%%%
\usepackage[T1]{fontenc}
\usepackage{hyperref}
\usepackage{color}
\usepackage[normalem]{ulem}
\usepackage{listings}
\usepackage{amsmath}
\usepackage{amssymb}
\lstset{inputpath=listings}
\renewcommand{\ttdefault}{pcr}
\lstdefinestyle{smallJava}{
  language=Java,
  basicstyle=\ttfamily\small,
  keywordstyle=\bfseries,
  numbers=left,
  stepnumber=1,
  numbersep=8pt,
  tabsize=2,
  showspaces=false,
  showstringspaces=false,
  frame=single,
  framexleftmargin=12pt,
  columns=fixed
}
\lstdefinestyle{refinity}{
  language=Java,
  basicstyle=\ttfamily\small,
  keywordstyle=\bfseries,
  alsoletter={\\},
  morekeywords={\\abstract\_statement},
  numbers=left,
  stepnumber=1,
  numbersep=8pt,
  tabsize=2,
  showspaces=false,
  showstringspaces=false,
  frame=single,
  framexleftmargin=12pt,
  columns=fixed
}

\usepackage{graphicx}
\usepackage{bussproofs}
\usepackage{subcaption}
\usepackage{xcolor}
\usepackage{tikz}
\usetikzlibrary{calc,positioning,arrows}
\captionsetup{compatibility=false}
\captionsetup[subfigure]{justification=centering}
%%%%%%%%%%%%%%%%%%%%%%%% fixme
\usepackage[draft,silent]{fixme}
\fxsetup{theme=color,mode=multiuser}
\definecolor{fxtarget}{rgb}{0.8000,0.0000,0.0000}
\FXRegisterAuthor{todo}{TODO}{TODO}
\FXRegisterAuthor{vp}{VP}{\color{violet}\scriptsize{\bf VP}}
\FXRegisterAuthor{vs}{VS}{\color{cyan}\scriptsize{\bf VS}}
\FXRegisterAuthor{oa}{OA}{\color{blue}\scriptsize{\bf OA}}
\FXRegisterAuthor{ek}{EK}{\color{brown}\scriptsize{\bf EK}}
%%%%%%%%%%%%%%%%%%%%%%%% fixme

\newcommand{\codein}[1]{\lstinline[style=refinity]|#1|}
\newcommand{\Refinity}{REFINITY}
\newcommand\relevant{$\mathrm{relevant}$}
\newcommand\assignable{$\mathrm{assignable}$}
\newcommand\accessible{$\mathrm{accessible}$}
\newcommand\keyrule[1]{\ensuremath{\mathrm{#1}}}
\newcommand\jcode[1]{\lstinline[style=smallJava]|#1|}
\newcommand\rcode[1]{\lstinline[style=refinity]|#1|}
\newcommand\jmcode[1]{\lstinline[mathescape=true,style=smallJava]|#1|}
\newcommand\rmcode[1]{\lstinline[mathescape=true,style=refinity]|#1|}
\newcommand\rname[1]{\textsc{#1}}
\newcommand\EK[1]{\textcolor{brown}{#1}}


\pagestyle{plain}

\title{Towards practical abstract execution}
\author{Ole J{\o}rgen Abusdal\inst{1} \and Eduard Kamburjan\inst{2} \and Violet Ka I Pun\inst{1} \and Volker Stolz\inst{1}}

\institute{%
Western Norway University of Applied Sciences, Norway\\
\email{\{ojab,vpu,vsto\}@hvl.no}
\and University of Oslo, Norway\\
\email{eduard@ifi.uio.no}
}

% What we want
% Talk about the renaming issue (identify in which cases it's interesting)
% Talk about relational verification
% For NPEs:
%  1. several ways to specify whether two objs. are equiv., need to decide that equiv!
%  2. Observables, if observ. are event-like then we can do it a history variable (as in KeY-ABS) otherwise we need symbolic-traces
% 3. For e.g. the array coordinate problem we need relational invariants in addition to relational postconditions (see mathias paper?)
% 


\begin{document}

\maketitle

\begin{abstract}
Relational verification through dynamic logic is a promising approach for verification of object oriented programs.
Recent advances from symbolic to abstract executions have enabled reasoning about incomplete/abstract versions of such programs.
This has proven fruitful in the exploration of correctness of refactorings primarily related to code blocks in Java.
In this paper we explore further types of equivalent transformations and refactorings and discuss the challenges that still
need to be overcome for full round-trip correctness of refactorings in object-oriented languages.
\end{abstract}

\section{Introduction}
Refactoring is a fundamental activity in software engineering to reorganize code to improve its structure, e.g., to simplify maintenance, while preserving its observable behavior of the program.
A refactoring can be defined as a pattern it matches on, and a subsequent program transformation on the matched part.
To ensure that the transformed program indeed has the same observable behavior one can either compare the transformed program with the 
original or %through, e.g., regression testing~\cite{regression} or relational verification~\cite{relate}, or 
reason about the program transformation itself.

Relational verification through dynamic logic is a promising approach to verification of refactoring patterns in object oriented programs.
Recent advances from symbolic to Abstract Execution (AE)~\cite{steinhoefel:ae} have enabled reasoning about incomplete/abstract versions of such programs.
This has proven fruitful in the exploration of correctness of refactorings primarily related to code blocks in Java:
%In this paper we explore further types of equivalent transformations and refactorings and discuss the challenges that still need to be overcome for full round-trip correctness of refactorings in object-oriented languages.
%
AE introduces abstract statements (and expressions), which act as named and specified placeholders for statement-sequences of the host language.
A refactoring proof is a relational verification proof that compares two programs which can have abstract program elements.
Consider relating
\begin{center}
\jmcode{if (E$_{\mathit{boolean}}$) \{S$_1$;\} else \{S$_2$;\} return;}\\[0.3em]
and\\[0.3em]
\jmcode{if (!E$_{\mathit{boolean}}$) \{S$_2$; return;\} S$_1$; return;}
\end{center}
where \jmcode{E$_{\mathit{boolean}}$} is an arbitrary boolean expression, \jmcode{S$_1$} and \jmcode{S$_2$} correspond to arbitrary statements.
One of the programs represents the schema of the code before the refactoring, and the other one the code afterwards, with respect to some relational post-condition that defines the notion of \emph{program equivalence}. This reasoning about abstract programs is exactly the aforementioned reasoning about the program transformation behind the refactoring.

Despite the promising results due to AE, it has only been applied to \emph{statement-level} refactorings that
change the body of a single method.
However, refactorings are not limited to single methods or statement-blocks, but may also restructure classes and data structures \cite{fowler:refactoring}.
In this paper we investigate the role of AE for verification of refactorings beyond statement-level.


\paragraph{Challenges.}
More complex refactorings require more elaborate specification and verification techniques for relational verification. 
The reason is that the programs surrounding the abstract statements, as well as the notion of equivalence, become more involved.
This holds for both structure and behaviour.
Following the general approach of AE, \Refinity{} \cite{steinhoefel:ae}, the tool supporting verification on top of the KeY system, an automated theorem prover for Java~\cite{DBLP:conf/aplas/Steinhofel20},
has been primarily designed to verify the correctness of refactorings that are based on moving code within a method, or on extracting some statements into their own method (i.e., the \vpnote*{cite?}{\rname{Extract Method} refactoring}).

To investigate the use of AE for refactorings on the class-level, we investigate the \vpnote*{cite?}{\rname{Hide Delegate} refactoring} that moves code between classes and show how it can be encoded in \Refinity{}.
As for specification, we discuss the interpretation of \textit{equivalence} from the perspective of the user and how to encode this --- for example, if the surrounding programs throw exceptions, under which conditions are the exceptions considered equivalent?
The equivalence interpretation goes beyond exceptions, but touches on a fundamental problem:
there are several possible choices for when newly created objects (and exceptions, which are objects in Java) are considered to be equal in relational verification using dynamic logic.
This was first investigated by Beckert et al.~\cite{DBLP:conf/lopstr/BeckertBKSSU13} and we discuss alternatives here.


For exceptions and object creation, we describe several possibilities when newly created objects (resp.\ thrown exceptions) are considered equal
and how this information can be used in dynamic logic proofs. The different possibilities are implemented as 
%\vsnote*{Unclear, at least that early in the paper. Why ``different''? Also see my later point about why using the new rule(s) should be optional.}{different rule sets} 
\EK{multiple (sets of) rules, from which the developer chooses the one corresponding to his assumptions on object allocations -- this choice
does not have to be encoded explicitly in the relational post-condition.}
This reduces the size of the required specification, which is an advantage, since it is a notorious bottleneck in formal verification~\cite{DBLP:journals/corr/abs-1211-6186,DBLP:series/lncs/HahnleH19}.


Furthermore, we discuss the necessary extensions needed to prove equivalent behaviour where one data structure is replaced by another, e.g., an array or any primitive type by a class.
Here, the main challenge lies in the encoding that the structures are used correctly throughout the execution.
This could, for example, be handled by coupled invariants~\cite{DBLP:conf/birthday/BeckertU18}, whose connection to AE is yet unclear. 
Lastly, we discuss the challenge to apply AE to novel specification approaches for \emph{traces} which aim to simplify the specification of temporal properties for expressive properties, but whose use for relational verification is unexplored.

\paragraph{Contributions and Structure.}
Our contributions include an investigation into the necessary side conditions to be able to proof of the correctness of the \rname{Extract Local Variable}- and \rname{Hide Delegate} refactorings, extending the collection of proven refactorings.
The latter is a refactoring beyond code motion within a method and highlights the interaction of AE with general relational verification challenges.
Then, we discuss possible extensions that would be required to address further refactorings with AE.

%This paper is structured as follows. 
We first describe AE and relational verification using the \rname{Extract Local Variable} refactoring in Section~\ref{sec:prelim}, before we show necessary conditions for the \rname{Hide Delegate} refactoring to be correct  and then discuss the challenges for AE in Section~\ref{sec:challenges}.
\vsnote*{@ALL}{Sec.~\ref{sec:future} proposes future directions for improvements in REFINITY and AE which most likely require major development effort.}
In Section~\ref{sec:discussion}, we discuss our results. We discuss the related work in  Section~\ref{sec:related}, and lastly conclude in Section~\ref{sec:conclusion}.



%This paper is structured as follows: we first provide an overview on abstract executions, the main vehicle for correctness- and equivalence proofs, and their use in REFINITY.
%Next, we provide an encoding of the Hide Delegating refactoring for REFINITY that moves away from statement-based refactorings to object-oriented refactorings in Section \ref{sec:hideDelegate}.
%After that in Section \ref{sec:challenges}, we explore (and address, where possible) \vsnote*{Not really limitations...better word?}{limitations} in REFINITY that we have discovered by investigating further refactorings.
%Our main goal is to extend the notion of \textit{equal} programs to \textit{equivalent} programs, e.g. by replacing data types.
%\vsnote{Mention possible application to optimizations already in intro? Probably.}
%We conclude with a discussion in Section \ref{sec:discussion} on the feasibility of necessary developments \vsnote*{esp. if we want to use histories in subsequent proofs.}{and their dependencies}.
% VS: This will then be included in the section above:
% Next we discuss issues related to formally correct and constructive handling of object creation to be able to express useful notions of equivalence in Section \ref{sec:objectcreation}.
% Section \ref{sec:trace} moves away from only heap-related equivalence and discusses tracking equivalent observable behaviour in a relational framework.
% The final challenge are relational invariants that we discuss in Section \ref{sec:relinv} which are required to track equivalent, yet structurally different representation of data types.

%%% Local Variables:
%%% mode: latex
%%% TeX-master: "main"
%%% End:


\section{Preliminaries: Abstract Execution}\label{sec:prelim}
\begin{itemize}
\item \vsnote*{}{Ole}
\item difference SE/AE
\item AE constructs (statements, expression, blocks, constraints)
\item short subsection on how R. works (lhs/rhs setup, generating PO
  for KeY, KeY proves mostly automatic).
\end{itemize}

\oanote*{find a way to cite the draft paper here p.7}{}

Succinctly, exactly as stated by Steinfhöfel in their Ph.D. thesis, “Abstract Execution” denotes the idea to process Abstract programs by symbolic Execution~\cite{steinhoefel-20}.
Symbolic execution (SE) \cite{DBLP:journals/csur/BaldoniCDDF18,DBLP:journals/ac/YangFBCW19} abstracts concrete execution by way of symbolic representations of language runtime
state in place of concrete machine representations of such artefacts.
Thus e.g. values, a store or a program counter all have a symbolic representation in SE.
Branching points, such as encountered when symbolically executing e.g. an if-then-else statement splits an execution path into new paths for each possible branch arm and for each such
path e.g. the symbolic store may be preserved in the new paths, but different conditions may also be carried through such that in the path where the symbolic program counter refers
to the then branch the evaluation of the boolean expression in the if statement must be valid whereas it is not valid in the path where the symbolic program counter refers to the else branch.
Possible executions are not just captured through branching paths, but along a path itself through the symbolic store; a symbolic value represents any valid concrete substitution.

\oanote*{find a way to mention Kripke structures in here, refer to KeY book p.55,fill in more concretely about what AE introduces over SE, use p.110 Dom's thesis}{}
The use of SE to potentially explore every possible execution a program can have is a popular program analysis technique (cite draft paper). AE extends SE by
introducing abstract program elements (APEs) to the base language that is symbolicaly executed.
For statements and expressions a corresponding abstract statement and abstract expression are introduced in addition; .
APEs represent any possible substitution with concrete program elements, be it statements or expressions, from the base language being symbolically executed.
Execution of APEs is then the leap taken in AE over traditional SE and require the introduction of abstract state changes, SE branching for
any abrupt completion an APE may have(early return, exceptions thrown,...), over-approximation of, a way to specifify the behavior of APEs ~\cite{steinhoefel-20}.

We introduce REFINITY by showing how to specify a method \rcode{void n()} where we do not want to specify the exact contents of its method body.
Only that it could possibly assign to some locations (its frame) and access some locations (its footprint).
Furthermore we wish to specify that no exceptions will be thrown by the method body.
We achieve such a specification in REFINITY by supplying it a definition of \rcode{n()} as seen in Listing \ref{lst:ExtractVariable-refinity-method}.
A method is specified and its only content is an abstract statement \rcode{\\abstract\_statement N;} preceeded by a comment where every line in the
comment begins with a \rcode{@}-character which denotes that this is a specification, and in this case a specification for the abstract statement.
The specification reads straigthforward with no suprises. An execution of the abstract statement \rcode{N} may have the effects of: assigning to the
abstract location set \rcode{frN}, accessing the abstract location set \rcode{fpN} and may not throw any exceptions.

An abstract location set represents a fixed set of memory locations a program may read or write to.
The set is fixed through a program's duration but the values at these locations may change.
When an abstract location set is placed in an assignable or accessible specification it is to be understood as an upper bound;
the locations may possibly all be accessed or assigned to, not at all or anything inbetween.

\begin{figure}[h]
  \centering
  \begin{minipage}{.65\linewidth}
  \lstinputlisting[linerange={2-8},style=refinity]{ExtractVariable/REFINITY/method.refinity}
  \captionof{lstlisting}{Method}
  \label{lst:ExtractVariable-refinity-method}
  \end{minipage}
\end{figure}


\begin{itemize}
\item in the following, we show a simple example for AE where we prove a new refactoring correct with REFINITY
\item with AE, we can specify a fully abstract method $n()$ (abstract body, explain locationsets, frame/footprint, global AE constraints -- do not specialise for Extract LV yet).
\item And now we can formulate the equivalence-question in Refinity with a before/after. Point out condition under which this is not correct,
\item the first attempt fails with open goals (give interpretation of open goal)
\item add AE-constraint to make the example go through
\item 1-2 sentences on ``instantiation-problem'', i.e. avoid giving specs that can't be implemented (example: assignable nothing) although technically Dom covered this already.
\end{itemize}

In the following, we show an example of how AE can be used to prove a refactoring correct with REFINITY. 

%%% Local Variables:
%%% mode: latex
%%% TeX-master: "main"
%%% End:


Let us consider the Extract Local Variable refactoring seen in Listing~\ref{lst:ExtractVariable-java}.
The example is an instance of a more general case, where preserving the behaviour of the program depends on other parts of the code.
The behaviour of the program changes if the method call \jcode{n()} has access to the attribute \jcode{x} and overwrites it.

Before we would have calls to \jmcode{o$_1$.n()} and then \jmcode{o$_2$.n()} whereas after applying the refactoring we would
have \jmcode{o$_1$.n()} followed by \jmcode{o$_1$.n()}. In this case if e.g. \jcode{n()} simply prints \jcode{this.toString()} we will
observe a difference in the two programs.

\begin{figure}[!h]
  \centering
  \begin{subfigure}{.2\linewidth}
    \lstinputlisting[style=refinity]{ExtractVariable/Java/before.java}
    \caption{Before}
  \end{subfigure}\hspace{1cm}
  \begin{subfigure}{.3\linewidth}
    \lstinputlisting[style=refinity]{ExtractVariable/Java/after.java}
    \caption{After}
  \end{subfigure}
\captionof{lstlisting}{Extract Local Variable}
\label{lst:ExtractVariable-java}
\end{figure}

The dynamic check for such a change detailed in other work~\ref{stolz:isolarefa} codified the necessity that the reference \jcode{x}
remains unchanged through the introduction of an assertion \jcode{assert temp == x;} to uncover violations after the fact,
which is useful e.g. when checking the refactored code against its suite of unit tests.

In Refinity one proceeds to verify a refactoring as correct by supplying the code before and after refactoring and supplying a desired precondition and postcondition to relate
the two programs. One essentially asks Refinity: Given these preconditions does the postcondition hold after abstract execution of these two programs? Refinity accepts a restricted
subset of Java, the same restrictions present in KeY, with extensions for abstract program fragments; pieces that are ``placeholders'' for concrete Java code.
The specification facilities for either Java constructs or abstact program fragments are described in the Java Modelling Language (JML) with some extensions.

We describe the refactoring to Refinity as shown with a left side (Before) in Listing~\ref{lst:ExtractVariable-refinity-before},
a right side (After) in Listing~\ref{lst:ExtractVariable-refinity-after} and a method level context (Method) in Listing~\ref{lst:ExtractVariable-refinity-method}.

Additionally some information is declared in parts of Refinity's interface that are not shown here:
Free program variables, here \rcode{x}; abstract location sets, here \rcode{frN} and \rcode{fpN}; relevant locations for the before and after code, here empty;
the desired precondition, here empty, and the desired postcondition.

The pre- and postcondition are specified in terms of equations that may relate the effects of the before and after side such as return values, exceptions thrown and any
of the relevant locations declared.
In our case we simply specify the postcondition that \rcode{x} returned in Before must be identical to \rcode{temp} returned in After and that any exception thrown must be identical for the two sides.

What can be seen in the first three lines of the Before and After sides is an \emph{abstract execution constraint} on \rcode{x}.
Namely that \rcode{x} should be disjoint from the abstract location set \rcode{frN}.

An abstract location set represents a fixed set of memory locations a program may read or write to. The set is fixed through a program's duration but the values at these locations may change.
Abstract location sets can be specified as assignable or accessible at various program locations as seen in Listing~\ref{lst:ExtractVariable-refinity-method}.
When a location set is placed in an assignable or accessible specification it is to be understood as an upper bound; the locations may possibly all be accessed or assigned to, not at all or
anything inbetween.

While inspecting the method \rcode{n()} you will discover an \emph{Abstract Statement} (AS) \rcode{\\abstract\_statement S;} immediately trailing the assignable and accessible specification for it.
This reprecent a ``placeholder'' for any statement that adheres to the specification given for it, here we only specified that whatever statement that executes there \emph{could} assign to the abstract
location set \rcode{frN} and \emph{could} access \rcode{fpN}.

What we have done by specifying the \emph{abstract execution constraint}, the first three lines in Before and After, is to place an additional restriction on the assignable abstract
location set \rcode{frN} of the AS present in \rcode{n()} stating that \rcode{x} is not an element of \rcode{frN}. As such \rcode{n()} should not be able to alter the value (a reference)
\rcode{x} holds.

We use an assertion to ensure we consider only the refactoring when \rcode{x} is an instance of its intended type.

\begin{figure}[!h]
  \centering
  \begin{subfigure}[b]{.34\linewidth}
    \lstinputlisting[style=refinity]{ExtractVariable/REFINITY/before.refinity}
    \caption{Before}
    \label{lst:ExtractVariable-refinity-before}
  \end{subfigure}\hspace{1cm}
  \begin{subfigure}[b]{.34\linewidth}
    \lstinputlisting[style=refinity]{ExtractVariable/REFINITY/after.refinity}
    \caption{After}
    \label{lst:ExtractVariable-refinity-after}
  \end{subfigure}\vspace{1mm}
  \begin{subfigure}[b]{.65\linewidth}
    \lstinputlisting[style=refinity]{ExtractVariable/REFINITY/method.refinity}
    \caption{Method}
    \label{lst:ExtractVariable-refinity-method}
  \end{subfigure}
\captionof{lstlisting}{Extract Local Variable}
\label{lst:ExtractVariable-refinity}
\end{figure}

Refinity manages to prove automatically that the postcondition holds after abstract execution of both side; here meaning that
\rcode{x} and \rcode{temp} hold the same reference and that any execptions thrown are identical.

One can consider what we have managed to encode and prove a partial success.
Indeed if we remove the constraint on \rcode{x} not being assigned to as a side-effect of executing \rcode{n()}  Refinity fails to prove our postcondition end up holding.
Yet this is a slightly unnatural encoding if we are to consider concrete instantiations of what we have expressed here.
We do not really intend to encode that return statements must occur at the end as shown, but that from that point onwards the reference \rcode{x} and \rcode{temp} hold
are identical. Thus we manage to encode our desired postcondition, but not capture the desired concrete instantiation with our fragments.

%%% Local Variables:
%%% mode: latex
%%% TeX-master: "../main"
%%% End:


\section{Challenges}\label{sec:challenges}
In this section we explore the possibilities of \Refinity{} beyond the original work.
In particular, we are interested in moving away from statement-based refactorings to more complex changes that also affect the structure of the code.
The first new refactoring, Hide Delegate, harnesses Java inner classes to express the desired behaviour of \Refinity{} in a straight-forward manner
and KeY automatically completes the proof once the right preconditions are identified.

After that, we present a refactoring that is related to Dead Code Elimination.
This requires introduction of additional constructs for object-construction within KeY,
which are then harnessed in tactlets that express that certain non-identical heaps guarantee equivalent behaviour.

We continue our wishlist for further flexibility in expressing equivalent program behaviour based on the execution history of user-defined observable actions,
and finally discuss challenges related to expressing equivalence in the face of different data types.

\subsection{Encoding the \rname{Hide Delegate} refactoring}\label{sec:hideDelegate}

The \rname{Hide Delegate} refactoring can be described as an \rname{Extract Method} refactoring on a call chain. Consider the statement \jcode{Y y = o.f().g()},
where the call chain is extracted to a new method on \jcode{o}, say \jcode{h()}, which contains the extraction \jcode{Y h() \{ return this.f().g(); \}}, such that we can replace the chain above with \jcode{Y y = o.h()}.
The refactoring can enable less coupling as the class that contained the call chain afterwards does not need to know the return type of \jcode{f()}.

\begin{figure}[tbp]
  \centering
  \begin{subfigure}{.3\linewidth}
    \begin{tikzpicture}[node distance=1cm,auto,>=stealth']
    \node[] (this) {\jcode{this}};
    \node[right = of this] (o) {\jcode{o}};
    \node[right = of o] (null) {\jcode{null}};
    
    \node[below of=this, node distance=3cm] (thisb) {};
    \node[below of=o, node distance=3cm] (ob) {};
    \node[below of=null, node distance=3cm] (nullb) {};

    \draw (this) -- (thisb);
    \draw (o) -- (ob);
    \draw (null) -- (nullb);

    \draw[->] ($(this)!0.25!(thisb)$) -- node[above,scale=1,pos=0.2]{\jcode{f()}} ($(o)!0.25!(ob)$);
    \draw[->] ($(this)!0.45!(thisb)$) -- node[above,scale=1,pos=0.1]{\jcode{g()}} node[below,scale=1,near start]{NPE} ($(null)!0.45!(nullb)$);
\end{tikzpicture}

    \vspace{-5mm}
    \caption{Before}
    \label{fig:hd-npe-before}
  \end{subfigure}
  \hspace{1.5cm}
  \begin{subfigure}{.3\linewidth}
    \begin{tikzpicture}[node distance=1cm,auto,>=stealth']
    \node[] (this) {\jcode{this}};
    \node[right = of this] (o) {\jcode{o}};
    \node[right = of o] (null) {\jcode{null}};
    
    \node[below of=this, node distance=3cm] (thisb) {};
    \node[below of=o, node distance=3cm] (ob) {};
    \node[below of=null, node distance=3cm] (nullb) {};

    \draw (this) -- (thisb);
    \draw (o) -- (ob);
    \draw (null) -- (nullb);
    \coordinate (loop) at ($(o)!0.35!(ob)$);

    \draw[->] ($(this)!0.25!(thisb)$) -- node[above,scale=1,pos=0.2]{\jcode{h()}} ($(o)!0.25!(ob)$);
    \draw[->,rounded corners=2pt] (loop) -- ($(loop) + (0.5,0) $) --  ($(loop) + (0.5,-0.1) $)  node[midway,scale=1]{\jcode{f()}} -- ($(loop) + (0,-0.1)$);    
    \draw[->] ($(o)!0.75!(ob)$) -- node[above,scale=1,pos=0.2]{\jcode{g()}} node[below,scale=1,midway]{NPE} ($(null)!0.75!(nullb)$);
\end{tikzpicture}

    \vspace{-5mm}
    \caption{After}
    \label{fig:hd-npe-after}    
  \end{subfigure}
  \caption{Null Pointer Exceptions (NPEs)}
  \label{fig:NPEs}
\end{figure}

Note that in the case of the more general pattern \jcode{X x = o.f(); Y y = x.g()} with non-interfering intermediate statements between the two statements,
we can reach the considered pattern through applications of \rname{Slide Statement} and finally \rname{Inline Variable} on \jcode{x}.



The scenarios shown in the sequence diagrams in Fig.~\ref{fig:NPEs} will both result in \jcode{NullPointerException} (NPE) when the call to \jcode{f()} returns \jcode{null}.
In the strictest sense of behavioral preservation,
we will observe a difference in the behaviour before and after the refactoring.  Concretely, the NPE thrown in Fig.~\ref{fig:hd-npe-before} (before) will show a different stacktrace than the one thrown in Fig.~\ref{fig:hd-npe-after} (after).
Thus we consider a behavioral equivalence that allows for disagreement in stacktraces for such matching exceptions. In fact one is unable to make any other distinction in \Refinity{} as it does not consider such effects.
%\vsnote{Something about post-condition missing here? I don't remember, does R. not actually model the stacktrace in an exc. anyways or did we need something explicit in the post-cond?}
% Ole: post-condition mention comes below the figure below

We specify the before- and after-program fragment for a \rname{Hide Delegate} refactoring in Listing~\ref{lst:HideDelegate-nofields-refinity} which faithfully captures the previously sketched out refactoring.
The classes and methods used in the refactoring are presented in Listing~\ref{lst:HideDelegate-nofields-classes-refinity} and show that we minimally specify the contents of the involved methods by using abstract statements in their bodies.
Note that we allow abrupt completion in the abstract statements \rcode{F} and \rcode{G} in the methods \rcode{getOwner()} and \rcode{getResource}.
That means the abstract statements may for instance throw exceptions.  For instance, the sketched out scenario considered in Fig.~\ref{fig:NPEs}, where \rcode{getResource()} will return \rcode{null} and cause the following call to throw a NPE, is a possibility.

\begin{figure}[tbp]
  \captionsetup{type=lstlisting}
  \centering
  \begin{sublstlisting}[b]{.455\linewidth}
    \lstinputlisting[style=refinity]{HideDelegate/REFINITY/before.refinity}
    \caption{Before}
    \label{lst:HideDelegate-nofields-before-refinity}
  \end{sublstlisting}\hspace{1cm}
  \begin{sublstlisting}[b]{.455\linewidth}
    \lstinputlisting[style=refinity]{HideDelegate/REFINITY/after.refinity}
    \caption{After}
    \label{lst:HideDelegate-nofields-after-refinity}
  \end{sublstlisting}
\caption{Program fragments for \rname{Hide Delegate} refactoring in REFINITY}
\label{lst:HideDelegate-nofields-refinity}
\end{figure}

\begin{figure}[tbp]
  \captionsetup{type=lstlisting}
  \centering
  \begin{sublstlisting}[b]{.4\linewidth}
    \lstinputlisting[style=refinity]{HideDelegate/REFINITY/Resource.refinity}
    \caption{Before}
    \label{lst:HideDelegate-nofields-resource-refinity}
  \end{sublstlisting}\hspace{1cm}
  \begin{sublstlisting}[b]{.4\linewidth}
    \lstinputlisting[style=refinity]{HideDelegate/REFINITY/Owner.refinity}
    \caption{After}
    \label{lst:HideDelegate-nofields-owner-refinity}
  \end{sublstlisting}
\caption{Classes in \rname{Hide Delegate} refactoring in REFINITY}
\label{lst:HideDelegate-nofields-classes-refinity}
\end{figure}


To prove the specified \rname{Hide Delegate} refactoring in the \vsnote*{Confusing, is this is ours or the historic version}{published} version (v0.9.7) of \Refinity{}, we need a postcondition that consists of a conjunction of return values of the before- and after-programs being identical and
that any thrown exceptions are both instances of NPE or otherwise equal.
This is owing to the fact that \Refinity{} does not consider occurrences of \jcode{new NullPointerException()}, or any other newly created objects, to be equal.
\vsnote{Is object-creation the only issue; see ``stacktraces'' above? I can't remember. In any, case it's probably JavaDL/KeY, not R's ``fault''.}

\OA{We have improved  \Refinity{}\footnote{Available at \footnotesize \url{https://github.com/selabhvl/REFINITY-abstractallocate}} to resolve this issue; we may keep the default postcondition that simply matches return results and exceptions, and
\Refinity{} automatically manages to prove the shown \rname{Hide Delegate} refactoring to be correct \vsnote*{@Ole: isn't that just the default one?}{wrt. the given postconditions}.
In the following section we will detail the changes needed to accomplish this.}

%%% Local Variables:
%%% mode: latex
%%% eval: (auto-fill-mode -1)
%%% TeX-master: "../main"
%%% End:

\subsection{Object creation}\label{sec:objectcreation}
\newcommand\relevant{$\mathrm{relevant}$}
\newcommand\assignable{$\mathrm{assignable}$}
\newcommand\accessible{$\mathrm{accessible}$}
\newcommand\keyrule[1]{\ensuremath{\mathrm{#1}}}

\begin{figure}
  \begin{subfigure}[h]{.45\linewidth}
    \lstinputlisting[style=refinity]{ObjectCreation/REFINITY/before.refinity}
    \caption{Before}
    \label{lst:ObjectCreation-refinity-before}    
  \end{subfigure}\hspace{1cm}
  \begin{subfigure}[h]{.45\linewidth}
    \lstinputlisting[style=refinity]{Objectcreation/REFINITY/after.refinity}
    \caption{After}
    \label{lst:ObjectCreation-refinity-after}
  \end{subfigure}
  \caption{Slide Object creation}
  \label{lst:ObjectCreation-refinity}
\end{figure}

As we have seen in the previous section, REFINITY encodes a rather harsh regimen on program equivalence:
in absence of a more fine-grained (application-specific) post-condition, it encodes that return values or exceptions must be identical on both sides,
as must be the objects in the \relevant{} location set (and the observables in this location set must be adequately specified).

This already creates the first hurdle for expressing useful notions of equivalences: as heap updates are symbolically encoded in the proofs, equivalent object-allocations are not directly identified by KeY as such.
Consider the following two pieces of code in Listing \ref{lst:ObjectCreation-refinity}
if the two allocations are independent from each other, i.e. there is no dependency in their constructors, such as a shared global counter,
both programs are equivalent, yet cannot be proved automatically due to syntactically different heaps.
To address this limitation, we need additional rules that allow ignoring heap updates that are irrelevant.
Surprisingly, the extant literature does not provide much automation there, but rather builds on restrictive specifications wrt.\ \assignable{}/\accessible{} frames.

Coming back to our example, it is syntactically obvious that both sides yield different heaps.
Before addressing the underlying problem, \vsnote{@Eduard: Let's skip the obvious that KeY didn't know that two objects allocated in identical heaps are hence identical, or?}
we can immediately contribute a small taclet that states that two objects are identical, if their arguments to the constructor are equal and if any heap updates between the two allocations do not afffect the allocation of the second object.
Similar to KeY's \keyrule{dropUpdate_2}, we can easily eliminate these operations on the heap on unrelated \textit{types} and obtain equivalent programs.
\vsnote*{TODO: Eduard to elaborate a bit here.}{Unrelated types are of course only a shortcut, dropUpdate2 is much more specific, VS thinks -- but I can't decode the side conditions on the rule}.

Let us close with the remark that due to the nature of taclets, we have not so much proven this refactoring to be correct,
but rather (only) moved this decision further down the chain: unlike in a formalization of an OO model from ground up e.g.\ in the Coq-theorem prover,
we do not have a way of proving the taclets correct (i.e.\ derive them as lemmas) within KeY for \textit{any} program.
\vsnote{Eduard, pease confirm this claim and maybe adjust formulation to what's palpatable to our audience.}

\subsection*{Related: Dead Code Elimination}
Similar considerations of equivalent heap manipulations need to be considered also in the are of optimizations.
Take for example the trivial fragment in Listing \ref{lst:xisnewxisnew}.
\begin{lstlisting}
  C x;
  x := new C();
  x := new C();
\end{lstlisting}
Again, without any sideeffects upon constructor invocation, the first update is of no consequence.
The additional rules above however will not yet be sufficient to prove that this version is equivalent to the version without the redundant object creation and assignment.
The location-set mechanism would still insist that the second object created  in the redundant version is a different object from the first (and only) object created in the optimized version.
On the one hand this can be addressed through a relaxed post-condition where we accept that we only need \textit{an} object of the right type and arguments, but it relies on the side-condition of the constructor not having side-effects, which we \vsnote*{TODO Eduard: is it?}{cannot easily encode as part of the post-condition}.
We face the same issue if we would want to encode the absence of side-effects as precondition to a taclet.

%%% Local Variables:
%%% mode: latex
%%% TeX-master: "main"
%%% End:


\subsection{Trace properties}\label{sec:traces}
In the previous section, we have established that relational verification using abstract execution
must take care of general effects of language semantics outside the abstract statements.
In this section, we illustrate a different obstacle to practical abstract execution, which touches on its core specification principles: sequences of side effects and events.

%In the previous section, we have established that syntactical equality of heaps is too strict as a precondition to equivalent (or rather: identical) behaviour.
%In practice, 
Following Fowler's persuasion of what constitutes correct refactorings, developers are content if refactored code gives the same observable behavior~\cite{fowler:refactoring2nd}.

This behavior is first and foremost encoded through unit-tests, but also on tests on side-effects and their order (e.g.\ output via \texttt{print}-statements).
Here we then have a much more relaxed setting where equivalence is decoupled from the fine-grained program semantics.
More realistically, one may want to establish that after refactoring, certain operations happened in the same order.

Abstract execution supports the specification of read and write events, via dynamic frames, but does not specify their order or give a general possibility to
specify the order of side-effects/events. The most straightforward approach is to use a special model variable to keep track of the events explicitly in a trace, 
and specify properties using the surrounding logic, in our case, JavaDL. This approach has been taken for dynamics logics (without investigating relation verification) in, e.g., ABSDL~\cite{DBLP:journals/jlp/DinO14} and for relational verification (albeit without using a dynamic logic) by, e.g., Barthe et al.~\cite{DBLP:conf/fmcad/BartheEGGKM19}.
This explicit encoding of user-defined execution histories has the advantage that it is not only useful for proofs, but can also directly be harnessed in concrete unit-tests where we can explicitly compare the recorded history of an earlier execution of a test with the history of the same test but on the refactored code base.

A main advantage of encoding the trace in the post state is that relational verification (either based on self-composition or its variants~\cite{DBLP:conf/spc/DarvasHS05,DBLP:conf/csfw/BartheDR04,DBLP:conf/fm/BartheCK11} or using the proof obligation described above),
can use the state logic to describe both states.
However, first-order specifications of temporal properties have been proven to be unwieldy, large and hard to understand. 
%This places even more burden on the developer, as they have to make sure that their chosen encoding of events in the data structure captures the salient part of equivalence.
%Clearly, this data structure is sensitive to order and any object identities very much in the same way as is the syntactic representation of heap manipulations.
This led to the development of other dynamic logics which interact with novel trace specifications, such as BPL~\cite{DBLP:conf/tableaux/Kamburjan19} and symbolic traces~\cite{DBLP:conf/tableaux/BubelDHN15}, where the post-condition of a modality is a \emph{trace formula} without quantifiers over indices, whose models are single traces.
It has neither been investigated how such logics can be used for relational verification, nor how abstract statements can be specified wrt.\ such trace properties.
%, as well as the
%integration of other, more standard trace specification into dynamic logic, such as LTL~\cite{beckert} or session types~\cite{ifm}.
%For none of these approaches, abstract statements have been investigated, but . %it is clear how abstract statement can be specified.

%We sketch the first steps towards a possible solution using a simplified version of local Session Types for Active Objects~\cite{DBLP:journals/jacm/HondaYC16,DBLP:journals/csur/BoerSHHRDJSKFY17,DBLP:conf/ifm/KamburjanC18}, which are an example for BPL specifications.

We concentrate on, simplified, symbolic traces here, which are defined by the grammar
\[\theta ::= \lceil \phi \rceil ~|~ \mathsf{call}(\mathtt{m}) ~|~ \mathbf{finite} ~|~ \theta \ast\!\ast \theta \]
where $\lceil \phi \rceil$ denotes the trace where the state formula $\phi$ holds, $\mathsf{call}(\mathtt{m})$ a call event on method $\mathtt{m}$, $\mathbf{finite}$ an arbitrary finite trace without any events\footnote{We deviate here from the original definition for example's sake.} and $\ast\ast$ is the chop~\cite{DBLP:conf/tableaux/BubelDHN15}, a special concatenation.
Models for such formulas are traces: sequences of states and events.

Consider the two programs in Listing~\ref{lst:sth}.
It shows a refactoring of some program using a \codein{File}, where we only give the trace specification.
The first program specifies that some initialization happens that is guaranteed to call \codein{f.open}, then the file is read and written,
and then some finalization happens that calls \codein{f.close}.
The second program switches the order of read and write.
The programs are, obviously, not equivalent in a strict sense, but if the trace property we are interested in is only concerned with the order of operations on the file 
(\codein{open}, \codein{write}, \codein{read} and \codein{open}), for example to express that only opened files are read and written to, then we need to specify a notion of equivalences.
It remains to be seen in how far abstract executions can be expanded with such a mechanism in the future.

\begin{figure}[tbp]
\captionsetup{type=lstlisting}
\centering
\begin{sublstlisting}{.8\linewidth}
\begin{lstlisting}[style=refinity]
File f = new File();
String s = "";
/*@  ensures finite ** call(f.open) ** finite; */
\abstract_statement A;     
s = f.read();
f.write(s);
/*@  ensures finite ** call(f.close) ** finite; */
\abstract_statement B;     
\end{lstlisting}
\caption{Before}
\end{sublstlisting}
\begin{sublstlisting}{.8\linewidth}\vspace{1mm}
\begin{lstlisting}[style=refinity]
File f = new File();
String s = "";
/*@  ensures finite ** call(f.open) ** finite; */
\abstract_statement A;     
f.write(s);
s = f.read();
/*@  ensures finite ** call(f.close) ** finite; */
\abstract_statement B;     
\end{lstlisting}
\caption{After}
\end{sublstlisting}
\caption{Refactoring for Trace-Based-Notions of Equivalence.}
\label{lst:sth}
\end{figure}


%\vsnote*{@ALL TODO: some conclusion here missing}{Speculate about difficult in reconciliating traces + AEs \textit{here}}.

%Harnessing output as a mechanism to decide equivalence requires instrumentation of both software versions, though:
%we need to extend our code with a (global) variable that accumulates observable behaviour at dedicated locations in the program.
%Making this value the return value on both sides allows us to leverage the colismponent-wise equality in the generic post-condition that REFINITY generates,
%and achieves the desired effect when we completely ignore differences in the relevant locations set by not including any variables in there at all.

%%% Local Variables:
%%% mode: latex
%%% eval: (auto-fill-mode -1)
%%% TeX-master: "main"
%%% End:


\subsection{Relational invariants}\label{sec:relational}
Another area of interest for equivalence is replacing one data structure with another,
e.g.\ Fowler's \rname{Replace Array with Object}~\cite[p.186]{fowler:refactoring} or \rname{Replace Primitive with Object}~\cite{fowler:refactoring2nd}.
As an example, in the following we look at a piece of code where an array is replaced with an object (or vice versa).
Again, from the strict default perspective of ``equal return values, equal heaps'', any two programs using the data structures are obviously not equal.
Encoding observability through traces as per the previous section will obviously solve this issue.
A new challenge arises when both programs use different or disjoint sets of operations, i.e.,\ we have different alphabets for their trace languages.

\begin{figure}[tbp]
  \captionsetup{type=lstlisting}
  \centering
  \begin{sublstlisting}[h]{.45\linewidth} 
    \lstinputlisting[style=smallJava]{ReplaceArrayWithObject/Java/before.java}
    \caption{Before}
    \label{lst:ReplaceArray-java-before}
  \end{sublstlisting}\hspace{1cm}
  \begin{sublstlisting}[h]{.45\linewidth}
    \lstinputlisting[style=smallJava]{ReplaceArrayWithObject/Java/after.java}
    \caption{After}
    \label{lst:ReplaceArray-java-after}
  \end{sublstlisting}
  \caption{Replace array with object refactoring}
  \label{lst:ReplaceArray-java}
\end{figure}

Let us consider a refactoring that replaces an array containing a geographical position given by a latitude and longitude as in Listing~\ref{lst:ReplaceArray-java-before} with an object that gives read or write access to the same values through setters and getters that make it immediately clear what is being accessed as seen in Listing~\ref{lst:ReplaceArray-java-after}.
In either direction of this refactoring, we must be certain that indexing can only occur within the bounds of the original program as there will be no corresponding out of bounds failure for method calls.
We note the added (syntactical) complication that in the direction from array to object, that if array offsets are computed, there is no direct correspondence to a setter/getter,
and the refactored code needs to dispatch on the corresponding component.
In the following, for simplification, we assume that the array is only used with constants.

% There may be usage sites in the original program which in some way rely on the array representation.
%These must be identified and handled in an equivalent way in the refactored program.


\begin{figure}[tbp]
\captionsetup{type=lstlisting}
\begin{sublstlisting}[b]{.45\linewidth} 
\begin{lstlisting}[style=smallJava]
String[] p = new String[2];
/* M only uses the static
method API to modify `p` and
does not assign a new value
to `p`. */
\abstract_statement M
setX(p,value);
\end{lstlisting}
\caption{Before}
\label{lst:ArrayBefore}
\end{sublstlisting}\hspace{1cm}
\begin{sublstlisting}[b]{.45\linewidth}
\begin{lstlisting}[style=smallJava]
Pos p = new Pos();
/* ditto */
\abstract_statement M
p.setX(value);
\end{lstlisting}
\caption{After}
\label{lst:ArrayAfter}
\end{sublstlisting}
\caption{Replace array with object refactoring}
\label{lst:ReplaceArray-REF}
\end{figure}

Having established that the two programs are not equal, but should be considered equivalent,
we need to establish a correspondence in the specific case.
To avoid having to establish the correspondence in all uses (of either setter/getter or the array),
we can assume that accesses in either case are wrapped in a method --- if we assume that \rname{Extract Method}/\rname{Inline Method} are already proven as correct.
This reliance on other refactorings allows us to compartmentalise the reasoning, and mostly focus on contracts for the involved methods.
While abstract execution is good at reasoning about placeholders for abstract behaviour, a similar mechanism for abstract structures is missing.
Through some intermediate steps, we can offload most of the reasoning to a history-based mechanism with some static assumptions on the code that can be easily checked.
The first is to encapsulate all accesses in methods; setters/getters on the object-side, and static helpers on the array-side.
We can then formulate the abstract programs for \Refinity{} as given in Listing.~\ref{lst:ReplaceArray-REF} for each of the operations in the API, here the matching pairs of setters/getters.
It remains to show that a) neither abstract statement overwrites \texttt{p}, that b) the histories of API operations with arguments called on either side match pairwise,
and that c) in the post-condition the values in their respective components match.

%\vsnote*{}{After spelling this out, it looks like we should just start with \href{https://memberservices.informit.com/my_account/webedition/9780135425664/html/replaceprimitivewithobject.html}{``Replace Data Value with Object''}.}

% \subsection*{Replacing interface-based implementations}

% \begin{figure}[tbp]
% %      \begin{lstlisting}[style=smallJava]
% %IFace p /* global free program variable */
% %post: true
%  %   \end{lstlisting}

%   \begin{subfigure}[h]{.45\linewidth} 
%     \begin{lstlisting}[style=smallJava]
% assert p instanceof ImplLHS;
% \abstract_statement M
%     \end{lstlisting}
%     \caption{Before}
%     \label{lst:IFaceBefore}
%   \end{subfigure}\hfill
%   \begin{subfigure}[h]{.45\linewidth}
%         \begin{lstlisting}[style=smallJava]
% assert p instanceof ImplRHS;
% \abstract_statement M
%     \end{lstlisting}
%     \caption{After}
%     \label{lst:iFaceAfter}
%   \end{subfigure}
%   \captionof{lstlisting}{Replace interface implementation}
%   \label{lst:refa-ibased}
% \end{figure}

% Changing implementations, i.e., replacing one class with another of the same interface,
% should ideally face no technical issues in a formal tool: we can declare a free variable of the corresponding type,
% and use it --- in contrast to the array/object-example above --- with \textit{identical abstract statements}.
% However, we need to rely on \textit{method contracts} for operations in the interface, and that the implementations actually respect this interface.
% Clearly the two underlying objects would not be identical, but we could formulate the post-condition guaranteeing equivalence by asserting component-wise equality.
% \vsnote*{}{Interesting issue with recursive classes like Lists...}
% Listing~\ref{lst:refa-ibased} illustrates the corresponding verification task for \Refinity{}.
% It remains to handle the relevant location set correctly, as nonetheless we are changing the object and hence have to restrict the types accordingly.
% \vsnote*{TODO: Ole/Eduard figure out/check specification of $M$ \& postcondition?}{}

% \Refinity{} currently does \vsnote*{TODO: Ole reporting current state of affairs here}{or does not} succeed in proving \vsnote*{Exactly what are we asking REFINITY, see above}{this}.

% \noindent\rule{\textwidth}{2pt}

%%%%%%%%%%%%%
% \vsnote*{}{Older text to be adapted/reused.}

% In REFINITY it would be difficult to even model this because what we would most naturally want to do is to somehow write a LHS with an abstract statement $P$ and a RHS with an abstract statemetn $Q$ and specify that $P$ contains an array used to index some data as seen in lis.~\ref{refa:ReplaceArray-java}. Then we would want to require somehow that $Q$ is exactly as $P$ but with the array substituted for a data object and any reads or writes to the array replaced with appropriate setters and getters. Such things are not possible as on the modeling level the abstract elements operate as black boxes whose contents we can't specify.

% If we attempt to specify the refactoring in REFINITY so that it captures all refactorings of the kind shown in lis.~\ref{refa:ReplaceArray-java} the closest we come is as show in lis.~\ref{refa:ReplaceArray-refinity} where valid instantiations
% also include programs that don't contain any arrays at all.

% \begin{figure}
%   \begin{subfigure}[b]{.45\linewidth} 
%     \lstinputlisting[style=refinity]{ReplaceArrayWithObject/REFINITY/before.refinity}
%     \caption{Before}
%     \label{refa:ReplaceArray-refinity-before}   
%   \end{subfigure}\hfill
%   \begin{subfigure}[b]{.45\linewidth}
%     \lstinputlisting[style=refinity]{ReplaceArrayWithObject/REFINITY/after.refinity}
%     \caption{After}
%     \label{refa:ReplaceArray-refinity--after}
%   \end{subfigure}
%   \captionof{lstlisting}{A too wide pattern}
%   \label{lst:ReplaceArray-refinity}
% \end{figure}

%%% Local Variables:
%%% mode: latex
%%% eval: (auto-fill-mode -1)
%%% TeX-master: "../main"
%%% End:



%%% Local Variables:
%%% mode: latex
%%% eval: (auto-fill-mode -1)
%%% TeX-master: "main"
%%% End:


\section{Discussion}\label{sec:discussion}
In the following, we shortly discuss some of the raised issues and in how far they can be adressed in the future and possibly in the short-term.

The first challenge is that due to the way \Refinity{} prepares the environment and the top-level proof obligation for KeY,
in fact both sides in a refactoring share the same Java namespace.
This means for example that the \rname{Extract Method} refactoring no so much proves the original correct, but rather a version where the extracted method is already present.
Care must be taken to set this up correctly, and e.g.\ in the \rname{Extract Method} example make sure that the methods is not used on the \textit{Before}-side.
The same holds for refactorings that remove code.
%\vsnote{Underwhelming.}

Adressing this would requiring choosing unique names in either schemata, since they go into a single KeY-proof, and hence would require some form of mapping classes/objects of distinct types (due to the nominal type system of Java) between both sides.
This issue is closely related to the challenge we discussed before when trying to relate unrelated yet semantically equivalent data types (see Section \ref{sec:relational}).
In general, we observe that currently \Refinity{} requires some scaffolding that makes the actual refactoring less obvious,
such as our use of assertions and casts.

A more general problem is capturing the most general instance of a refactoring.
Currently, \Refinity{} lacks placeholders for names, which means that e.g.\ in the \rname{Extract Local Variable} refactoring we would have to instantiate the \textit{Before}-schema for every concrete instance with the corresponding identifier names for variables.
The otherwise straight-forward refactoring \rname{Rename Temporary} already exposes the issue faced by \Refinity{} due to renaming on the block-level.
%\vsnote*{We didn't spell it out, but maybe EK can confirm that this would be an issue for/with R. and delete this comment.}
Likewise, the related \rname{Hide Delegate} example uses concrete method- and class names.
During our development of this refactoring, we have found us revising the encoding and use of placeholders with their annotations repeatedly.
Conversely, due to the challenges in checking instantiation of schemata against concrete programs already pointed out by Steinhöfel~\cite[119,137]{steinhoefel-20},
\vsnote{Maybe add page number -- or is this also mentioned in the paper. Ole?}
\oanote{instance checking is mentioned as an issue in several places, perhaps p119, p137 are most relevant}
one has to take care not to write too restrictive programs that rule out useful working instances

Another area for placeholders would be a generalisation of storage locations:
we foresee that there exist refactorings that may need to be specified twice, once with using local variables and another time using attributes in their schemata.
%\vsnote{@EK: Can you confirm that I'm making sense here? It turns out that none of our refactorings is actually an example for that.}
%\vsnote*{}{Eduard, are variables on the top-level of the method-context (and hence in Problem.java) equivalent to \textit{free program vars} in the sense that KeY always splits accesses into null/not-null when there is no concrete value? That was a bit unclear to us, since although in principle an AS could update them with a random value, we didn't see this explicitly in the proofs, or we missed it.}

%\vsnote*{}{Not sure if the following should be here or in future work -- which we don't have as of yet?}
For the time being we are limited to checking schemata against each other.
In the future, when we move on to checking instantiations, 
we feel that often necessary pre-conditions on a refactoring can easily be discharged by simple syntactical or static analysis (e.g. ``code does never read attribute \lstinline|x| of objects of type \lstinline|C|'').
Yet unlike in other formal work where the program is encoded as part of the proof-term, we cannot implement such analyses within KeY, and can only informally document and require such side-conditions on refactorings.
Correspondingly, we will also not be able to use KeY to formulate and prove a lemma that such a static property entails the necessary consequences.
%\vsnote*{}{Note to Eduard: what I mean here is that e.g. in Coq, if I would have the Java AST, I could implement \textbf{AND} prove as correct a static analysis. Can you add a sentence conjecturing on what would be possible if the code would be available as part of the proof-context? Or maybe it already is in some way?}


%%% Local Variables:
%%% mode: latex
%%% eval: (auto-fill-mode -1)
%%% TeX-master: "main"
%%% End:


\section{Related Work}\label{sec:related}

Similar or other approaches to formal verification of refactorings can be found in work by
Garrido et. al. \cite{garrido2006formal} who formalize \emph{``Push Down Method''},
\emph{``Pull Up Field''} and \emph{``Rename Temporary''} using an executable
Java formal semantics in Maude and give partially mechanized proofs for the two former.

Long Quan et. al. \cite{DBLP:conf/isola/QuanQL08} formulate refactorings as refinement
laws in the calculus of refinement of component and object-oriented systems (rCOS),
focusing on correctnes proofs of refactoring rules themselves.

While KeY and REFINITY is unique for its relational verification capacity for
schematic programs (or abstract programs) it is limited in power for verification
of concrete programs relying much more on manual specification or interaction \cite{DBLP:conf/aplas/Steinhofel20}
than tools like LLRêve \cite{DBLP:journals/jar/KieferKU18} or SymDiff \cite{DBLP:conf/cav/LahiriHKR12}
which offer more automation for concrete programs.





\section{Conclusion}\label{sec:conclusion}

We have presented two new encodings of refactorings (\rname{Extract Local Variable} and \rname{Hide Delegate}) and their necessary preconditions (constraints) for them to be behaviour-preserving for \Refinity{}.
\Refinity{} has syntactical constructs that capture abstract program executions, for which the KeY system,
an automated theorem prover for JavaDL, succeeds in proving the refactorings as correct (equal) wrt.\ to the Java semantics without user interaction.

The \rname{Hide Delegate} refactoring departs from statement-based refactorings and considers changes involving multiple classes and requires us to consider the first subtle difference between equivalent-yet-not-equal objects in the form of equivalent exceptions and how we need to explicitly address this in the post-condition of the proof-obligation.
This also allows us to make a contribution through further taclets that capture some indistinguishable programs that only differ in placement of objects on the heap.

We discuss in how far \Refinity{} could be used to capture other refactorings, broadening our investigation into the underlying notion of sometimes use-case specific \textit{equivalent behaviour}.
For example, while traces over observable behaviour could be explicitly encoded and checked against each other as return values in \Refinity{}, a more general process-algebra inspired approach of execution histories for abstract executions would avoid confounding the original program logic with scaffolding to achieve an encoding without having to extend the tool.
We also point out the difficulties due to a naming issue in the current encoding from Refinity to KeY proof-obligations for refactorings that change the class hierarchies or in general attempt to relate behaviour across different types.

\paragraph{Future Work.} 
To investigate the discussed problems with refactoring using AE wrt.\ trace properties, we are currently implementing AE for BPL in the Crowbar tool~\cite{crowbar} as a starting point, a symbolic execution engine to prototype behavioral symbolic execution.

We are also working on contributing further encodings of common refactorings that can already now be handled by \Refinity{}.
In addition, we are particularly interested in additional taclets for KeY that would enable automation of proofs
that currently are stuck on the explicit symbolic encoding of program state although it would be indistinguishable from equivalent states in pratice.
The latter is of relevance e.g. for proving common optimizations as in our \rname{Dead Code Elimination} example.

\subsubsection*{Acknowledgements}
This work was partially supported by the Research Council of Norway via \texttt{SIRIUS} (237898), \texttt{PeTWIN} (294600) and \texttt{CROFLOW} (326249).
%\input{conclusion}

% \section{Leftovers}
% \textit{Use if space/time/useful.}
% \subsection{Remove assignment to parameter}
% Let us consider a refactoring that eliminates assignments to method parameters as seen in fig.~\ref{refa:RemoveAssignment-java-before} and instead introduces a temporary variable initialized to the method parameter's value that replaces the usage of the method parameter at any usage site as seen in fig.~\ref{refa:RemoveAssignment-java-after}.

\begin{figure}
  \begin{subfigure}[h]{.45\linewidth}
    \lstinputlisting[style=smallJava]{RemoveAssignmentToParameter/Java/before.java}
    \caption{Before}
    \label{refa:RemoveAssignment-java-before}    
  \end{subfigure}\hspace{1cm}
  \begin{subfigure}[h]{.45\linewidth}
    \lstinputlisting[style=smallJava]{RemoveAssignmentToParameter/Java/after.java}
    \caption{After}
    \label{refa:RemoveAssignment-java-after}
  \end{subfigure}
  \caption{Replace array with object refactoring}
  \label{refa:RemovAssignment-java}
\end{figure}

In an attempt to specify this in REFINITY seen in fig~\ref{refa:RemovAssignment-refinity} we again fall short at the modeling level as we are not able to relate the abstract statement $Q$ in fig.~\ref{refa:RemoveAssignment-refinity-after} with the abstract statement $P$ in fig.~\ref{refa:RemoveAssignment-refinity-before} beyond that before $P$ assigned to the method parameter, stated by the \lstinline[style=refinity]|\hasTo(mParam)|, while after $Q$ assigns to a new variable, stated by \lstinline[style=refinity]|\hasTo(in)|. Additionally, by the abstract execution constraint seen in the very beginning, we state that the locations
assigned to \lstinline[style=refinity]|frP| and \lstinline[style=refinity]|mParam| are disjoint to be able to remove the method parameter \lstinline[style=refinity]|mParam| as an assignable location in the after case.

Naturally REFINITY is unable to prove fig~\ref{refa:RemoveAssignment-refinity-before} and fig~\ref{refa:RemoveAssignment-refinity-after} as equivalent
as we are not able to further relate $P$ and $Q$; we would need to somehow additionally state that $Q$ is produced through a substitution on $P$.


\begin{figure}
  \begin{subfigure}[h]{.45\linewidth}
    \lstinputlisting[style=refinity]{RemoveAssignmentToParameter/REFINITY/before.refinity}
    \caption{Before}
    \label{refa:RemoveAssignment-refinity-before}    
  \end{subfigure}\hspace{1cm}
  \begin{subfigure}[h]{.45\linewidth}
    \lstinputlisting[style=refinity]{RemoveAssignmentToParameter/REFINITY/after.refinity}
    \caption{After}
    \label{refa:RemoveAssignment-refinity-after}
  \end{subfigure}
  \caption{Replace array with object refactoring}
  \label{refa:RemovAssignment-refinity}
\end{figure}


\bibliographystyle{splncs04}
\bibliography{refs}

\end{document}

%%% Local Variables:
%%% mode: latex
%%% eval: (auto-fill-mode -1)
%%% TeX-master: t
%%% End:
