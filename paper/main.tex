\documentclass[runningheads]{llncs}


%%%%%%%%%%%%%%%%%
% USED PACKAGES %
%%%%%%%%%%%%%%%%%
\usepackage[T1]{fontenc}
\usepackage{color}
\usepackage{listings}
\lstset{inputpath=listings,numberbychapter=false}
\renewcommand{\ttdefault}{pcr}
\renewcommand{\lstlistingname}{Lis.}
\lstdefinestyle{smallJava}{
  language=Java,
  basicstyle=\ttfamily\small,
  keywordstyle=\bfseries,
  numbers=none,
  stepnumber=1,
  numbersep=8pt,
  tabsize=2,
  showspaces=false,
  showstringspaces=false,
  frame=single,
  columns=fixed,
  captionpos=b,
}
\lstdefinestyle{refinity}{
  language=Java,
  basicstyle=\ttfamily\small,
  keywordstyle=\bfseries,
  alsoletter={\\},
  morekeywords={\\abstract\_statement},
  numbers=none,
  stepnumber=1,
  numbersep=8pt,
  tabsize=2,
  showspaces=false,
  showstringspaces=false,
  frame=single,
  columns=fixed,
  captionpos=b
}

\usepackage{graphicx}
\usepackage{subcaption}
\captionsetup{compatibility=false}

%TODO add packages for this if needed
% \def\myttsize{\fontsize{9}{9.5}}
\def\codesize{\fontsize{10}{11}}
\def\rnsize{\fontsize{9}{9}}

\makeatletter
\newcommand{\onelettername}[1]{#1\aftergroup\@gobble}
\makeatother

\newcommand{\red}[1] {\textcolor{red}{#1}}
\newcommand{\blue}[1] {\textcolor{blue}{#1}}
\newcommand{\green}[1] {\textcolor{green}{#1}}
\definecolor{lightgray}{gray}{0.9}

\newcommand\object[1]{\Abs{#1}}
\newcommand\this{\texttt{this}}
\newcommand\rf[1]{\textsl{#1}}
% \newcommand\rff[2]{\textsl{#1 (#2)}}
\newcommand\rff[2]{\textsl{#1}}

\newcommand\srule[1]{\textsc[#1]}
\newcommand\oprule[1]{{\rnsize\selectfont\textsc{#1}}}
\newcommand\set[1]{\{#1\}}
\newcommand\config{\ensuremath{\mathit{cn}}}
\newcommand\fut[2]{\mathit{fut(#1,#2)}}
\newcommand\ob[4]{\mathit{ob(#1,#2,#3,#4)}}
\newcommand\process[2]{\{#1|#2\}}
\newcommand\invoc[4]{\mathit{invoc(#1,#2,#3,#4)}}
\newcommand\cog[2]{\mathit{cog(#1,#2)}}
\newcommand\cont[1]{\textsf{\textbf{cont}}\mathit{(#1)}}
\newcommand\error{\textsf{\textbf{error}}}
\newcommand\idle{\textsf{\textbf{idle}}}
\newcommand\eval[3]{[\![#1]\!]_{#2}^{#3}}
\newcommand\dom[1]{\text{dom}(#1)}
\newcommand\without{\backslash}
\newcommand{\many}[1]{\overline{#1}}
% \newcommand\obj[4]{\textit{ob}(\textit{#1},\textit{#2},#3,\textit{#4})}

\newcommand\rfeq{\equiv_\mathcal{R}}
\newcommand\refactoring{\ensuremath{\mathcal{R}\mathit{f}}}
\newcommand\refactor[1]{\ensuremath{\refactoring(#1)}}
% \newcommand\Aset{\mathcal{A}}
% \newcommand\Lset{\mathcal{L}}
% \newcommand\Sset{\mathcal{L}}
\newcommand\kw[1]{\textsf{\textbf{#1}}}
\newcommand\rl{\mathit{rn}}

%%%%%%%% Rules %%%%%%%%%%%

\newcommand{\ncondrule}[3]{ 
  \begin{array}{c} 
    \textsc{ ({#1})} \\[1pt] 
    #2 \\[1pt] 
    \hline\\[-7pt]
    #3 
  \end{array} }
 
\newcommand{\nrule}[2]{ 
  \begin{array}{c} 
    \textsc{ ({#1})} \\ 
    #2  
  \end{array}} 

%%%%%%%% End of Rules %%%%%%%%%%%

%%%% column type %%%%%
\newcolumntype{L}[1]{>{\raggedright\let\newline\\\arraybackslash\hspace{0pt}}m{#1}}
\newcolumntype{C}[1]{>{\centering\let\newline\\\arraybackslash\hspace{0pt}}m{#1}}
\newcolumntype{R}[1]{>{\raggedleft\let\newline\\\arraybackslash\hspace{0pt}}m{#1}}

\newcommand{\tool}[1]{\texttt{#1}}% some executable tool-name

\definecolor{brilliantlavender}{rgb}{0.96, 0.73, 1.0}
\definecolor{airforceblue}{rgb}{0.36, 0.54, 0.66}


\newcommand{\hlbox}[3]{\fboxsep=#1\colorbox{brilliantlavender!30}{#2#3}%
}
\newcommand{\cbox}[1]{\colorbox{brilliantlavender!30}{#1}%
}

% \framebox(180,9){some text (or keep it empty)}

%%% Local Variables: 
%%% mode: latex
%%% TeX-master: ""
%%% End: 

\pagestyle{plain}

\title{Towards practical abstract execution}
\author{Ole J{\o}rgen Abusdal\inst{1} \and Eduard Kamburjan\inst{2} \and Violet Ka I Pun\inst{1} \and Volker Stolz\inst{1}}

\institute{%
Western Norway University of Applied Sciences, Norway\\
\email{\{ojab,vpu,vsto\}@hvl.no}
\and University of Oslo, Norway\\
\email{eduard@ifi.uio.no}
}

% What we want
% Talk about the renaming issue (identify in which cases it's interesting)
% Talk about relational verification
% For NPEs:
%  1. several ways to specify whether two objs. are equiv., need to decide that equiv!
%  2. Observables, if observ. are event-like then we can do it a history variable (as in KeY-ABS) otherwise we need symbolic-traces
% 3. For e.g. the array coordinate problem we need relational invariants in addition to relational postconditions (see mathias paper?)
% 


\begin{document}

\maketitle

\begin{abstract}
 
\end{abstract}

\section{Introduction}
%
Relational verification through dynamic logic is a promising approach for verification of object oriented programs.
Recent advances from symbolic to abstract executions have enabled reasoning about incomplete/abstract versions of such programs.
This has proven fruitful in the exploration of correctness of refactorings primarily related to code blocks in Java.
In this paper we explore further types of equivalent transformations and refactorings and discuss the challenges that still need to be overcome for full round-trip correctness of refactorings in object-oriented languages.

This paper is structured as follows: we first provide an overview on abstract executions, the main vehicle for correctness- and equivalence proofs.
Next we discuss issues related to formally correct and constructive handling of object creation to be able to express useful notions of equivalence in Section \ref{sec:objectcreation}.

Section \ref{sec:trace} moves away from only heap-related equivalence and discusses tracking equivalent observable behaviour in a relational framework.
The final challenge are relational invariants that we discuss in Section \ref{sec:relinv} which are required to track equivalent, yet structurally different representation of data structures.

%%% Local Variables:
%%% mode: latex
%%% TeX-master: "main"
%%% End:


\section{Preliminaries: Abstract Execution}

\section{Object creation}

\section{Trace properties}

\section{Relational invariants}

\section{Discussion}
In this section we will discuss some of the limitations of REFINITY restricting the kinds of refactorings one can prove.

Consider a refactoring that replaces an array containing latitude and longitude positions as in fig.~\ref{refa:ReplaceArray-before} with an object that ``indexes'' the same values through setters and getters s.t. what the ``index'' is ``indexing'' is immediately clear as seen in fig.~\ref{refa:ReplaceArray-after}.

For this to be a refactoring we must be certain that indexing only can occur in bounds for the original program as there will be no corresponding out of bounds failure for method calls. There may be usage sites in the original program which in some way rely on the array representation, these must be identified and handled in an equivalent way in the refactored program.

\begin{figure}
      \label{refa:ReplaceArray}
  \begin{subfigure}[h]{.45\linewidth}
    \label{refa:ReplaceArray-before}    
    \lstinputlisting[style=smallJava]{ReplaceArrayWithObject/before.java}
    \caption{Before}
  \end{subfigure}\hspace{1cm}
  \begin{subfigure}[h]{.45\linewidth}
    \label{refa:ReplaceArray-after}
    \lstinputlisting[style=smallJava]{ReplaceArrayWithObject/after.java}
    \caption{After}
  \end{subfigure}
  \caption{Replace array with object refactoring}
\end{figure}

As it is there is no way in REFINITY to specify that given two abstract program fragments $P$ and $Q$ where $P$ contains
an array used to index some data as seen in fig.~\ref{refa:ReplaceArray} then $Q$ is exactly as $P$ but with
the array substituted for a data object and any indexing or assignments replaced with appropriate setters
and getters. 

In this section we will discuss two refactorings that we have found fitting for REFINITY's proving capabilites.

\section{Extract Variable}
Let us consider the Extract Local Variable refactoring which has had dynamic detection considered \ref{stolz:isolarefa} for one of its pitfalls.

TODO: Add motivating java code & dynamic fix

\begin{figure}
  \centering
  \begin{subfigure}{.4\linewidth}
    \lstinputlisting[style=refinity]{ExtractVariable/REFINITY/before.refinity}
    \caption{Before}
  \end{subfigure}\hspace{1cm}
  \begin{subfigure}{.4\linewidth}
    \lstinputlisting[style=refinity]{ExtractVariable/REFINITY/after.refinity}
    \caption{After}
  \end{subfigure}
\captionof{lstlisting}{Extract Local Variable}
\label{lst:ExtractVariable-refinity}
\end{figure}


\section{Hide Delegate}
\subsection{Encoding the Hide Delegate refactoring}\label{sec:hideDelegate}

The Hide Delegate refactoring can be described as an Extract Method refactoring on a call chain such as \jcode{Y y = o.f().g()}
the call chain is extracted to a method on \jcode{o} such that we can replace the chain with a call to the new method \jcode{o.h()}
which we declare as \lstinline|Y h() { return this.f().g(); }|.
\vsnote{I fixed a ``broken'' macro issue here.}
The refactoring can enable less
coupling as the class that contained the call chain afterwards does not need to know the return type of \jcode{f()}.
We note that this is just one example and intermediate, dotted, expressions can be arbitrary long and/or broken up across assignments to local variables.
\vsnote{Q: Generalize to \jcode{o = AE; z = o.g(); return z;} ?}
\vsnote{Three things on the nature of encoding things in R.: i) this is an example that R. is bad at \textit{names}: we need concrete names \texttt{f/g} (with concrete parameters), but want placeholders. ii) we don't really remove/replace anything, since we need all the code that we want to use in the same class, i.e., the ``new'' method \jcode{hideDelegate()} was already present before -- but that's okay because it's obviously correct to introduce methods that are not called -- another example of syntactical requirements on encodings in R. iii) This is the example for our abuse of R.\ for OO-based refactorings.}

\ref{lst:HideDelegate-nofields-refinity}


\begin{figure}
  \centering
  \begin{subfigure}[b]{.4\linewidth}
    \lstinputlisting[style=refinity]{HideDelegate/REFINITY/before.refinity}
    \caption{Before}
    \label{lst:HideDelegate-nofields-before-refinity}
  \end{subfigure}\hspace{1cm}
  \begin{subfigure}[b]{.4\linewidth}
    \lstinputlisting[style=refinity]{HideDelegate/REFINITY/after.refinity}
    \caption{After}
    \label{lst:HideDelegate-nofields-after-refinity}
  \end{subfigure}
\captionof{lstlisting}{Hide Delegate}
\label{lst:HideDelegate-nofields-refinity}
\end{figure}

\begin{figure}
  \centering
  \begin{subfigure}[b]{.4\linewidth}
    \lstinputlisting[style=refinity]{HideDelegate/REFINITY/Resource.refinity}
    \caption{Before}
    \label{lst:HideDelegate-nofields-resource-refinity}
  \end{subfigure}\hspace{1cm}
  \begin{subfigure}[b]{.4\linewidth}
    \lstinputlisting[style=refinity]{HideDelegate/REFINITY/Owner.refinity}
    \caption{After}
    \label{lst:HideDelegate-nofields-owner-refinity}
  \end{subfigure}
\captionof{lstlisting}{Hide Delegate Classes}
\label{lst:HideDelegate-nofields-classes-refinity}
\end{figure}

\subsection*{Hide Delegate --- once more with fields}

Similarily a minor variation of the  Hide Delegate refactoring also applies to a call chain which assigns to temporaries or fields
e.g. \lstinline[style=smallJava]|x = o.f(); Y y = x.g()|.
\vsnote*{Probably not true.}{%
Here we must beware that in our replacement we may observe a difference
if \lstinline[style=smallJava]|g()| can leave \lstinline[style=smallJava]|x| in an altered state which is then eliminated by applying
the refactoring \lstinline[style=smallJava]|Y y = h();|
}



%TODO this is not yet done

%%% Local Variables:
%%% mode: latex
%%% eval: (auto-fill-mode -1)
%%% TeX-master: "../main"
%%% End:



\section{Related Work}
% Similar or other approaches to formal verification of refactorings can be found in work by
Garrido et. al. \cite{garrido2006formal} who formalize \emph{``Push Down Method''},
\emph{``Pull Up Field''} and \emph{``Rename Temporary''} using an executable
Java formal semantics in Maude and give partially mechanized proofs for the two former.

Long Quan et. al. \cite{DBLP:conf/isola/QuanQL08} formulate refactorings as refinement
laws in the calculus of refinement of component and object-oriented systems (rCOS),
focusing on correctnes proofs of refactoring rules themselves.

While KeY and REFINITY is unique for its relational verification capacity for
schematic programs (or abstract programs) it is limited in power for verification
of concrete programs relying much more on manual specification or interaction \cite{DBLP:conf/aplas/Steinhofel20}
than tools like LLRêve \cite{DBLP:journals/jar/KieferKU18} or SymDiff \cite{DBLP:conf/cav/LahiriHKR12}
which offer more automation for concrete programs.





\begin{itemize}
\item formal proofs of correctness for OO program optimisation
\item formal proofs of OO refactoring
\item check Dominic’s thesis and papers for the relevant related work
\item important: correctness on actual Java, not something which looks like Java
\item also: correctness with respect to OO features
\end{itemize}

\section{Conclusion}

%\input{conclusion}


\bibliographystyle{splncs04}
\bibliography{refs}

\end{document}

%%% Local Variables:
%%% mode: latex
%%% TeX-master: t
%%% End:
