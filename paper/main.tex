\documentclass[runningheads]{llncs}


%%%%%%%%%%%%%%%%%
% USED PACKAGES %
%%%%%%%%%%%%%%%%%
\usepackage[T1]{fontenc}
\usepackage{hyperref}
\usepackage{color}
\usepackage[normalem]{ulem}
\usepackage{listings}
\usepackage{amsmath}
\usepackage{amssymb}
\lstset{inputpath=listings,numberbychapter=false}
\renewcommand{\ttdefault}{pcr}
\renewcommand{\lstlistingname}{Lis.}
\lstdefinestyle{smallJava}{
  language=Java,
  basicstyle=\ttfamily\small,
  keywordstyle=\bfseries,
  numbers=none,
  stepnumber=1,
  numbersep=8pt,
  tabsize=2,
  showspaces=false,
  showstringspaces=false,
  frame=single,
  columns=fixed,
  captionpos=b,
}
\lstdefinestyle{refinity}{
  language=Java,
  basicstyle=\ttfamily\small,
  keywordstyle=\bfseries,
  alsoletter={\\},
  morekeywords={\\abstract\_statement},
  numbers=none,
  stepnumber=1,
  numbersep=8pt,
  tabsize=2,
  showspaces=false,
  showstringspaces=false,
  frame=single,
  columns=fixed,
  captionpos=b
}

\usepackage{graphicx}
\usepackage{bussproofs}
\usepackage{subcaption}
\usepackage{xcolor}
\captionsetup{compatibility=false}
\captionsetup[subfigure]{justification=centering}
%%%%%%%%%%%%%%%%%%%%%%%% fixme
\usepackage[draft,silent]{fixme}
\fxsetup{theme=color,mode=multiuser}
\definecolor{fxtarget}{rgb}{0.8000,0.0000,0.0000}
\FXRegisterAuthor{todo}{TODO}{TODO}
\FXRegisterAuthor{vp}{VP}{\color{violet}\scriptsize{\bf VP}}
\FXRegisterAuthor{vs}{VS}{\color{cyan}\scriptsize{\bf VS}}
\FXRegisterAuthor{oa}{OA}{\color{blue}\scriptsize{\bf OA}}
\FXRegisterAuthor{ek}{EK}{\color{brown}\scriptsize{\bf EK}}
%%%%%%%%%%%%%%%%%%%%%%%% fixme

\newcommand{\codein}[1]{\lstinline[style=refinity]|#1|}
\newcommand{\Refinity}{REFINITY}
\newcommand\relevant{$\mathrm{relevant}$}
\newcommand\assignable{$\mathrm{assignable}$}
\newcommand\accessible{$\mathrm{accessible}$}
\newcommand\keyrule[1]{\ensuremath{\mathrm{#1}}}
\newcommand\jcode[1]{\lstinline[style=smallJava]|#1|}
\newcommand\rcode[1]{\lstinline[style=refinity]|#1|}
\newcommand\jmcode[1]{\lstinline[mathescape=true,style=smallJava]|#1|}
\newcommand\rmcode[1]{\lstinline[mathescape=true,style=refinity]|#1|}


\pagestyle{plain}

\title{Towards practical abstract execution}
\author{Ole J{\o}rgen Abusdal\inst{1} \and Eduard Kamburjan\inst{2} \and Violet Ka I Pun\inst{1} \and Volker Stolz\inst{1}}

\institute{%
Western Norway University of Applied Sciences, Norway\\
\email{\{ojab,vpu,vsto\}@hvl.no}
\and University of Oslo, Norway\\
\email{eduard@ifi.uio.no}
}

% What we want
% Talk about the renaming issue (identify in which cases it's interesting)
% Talk about relational verification
% For NPEs:
%  1. several ways to specify whether two objs. are equiv., need to decide that equiv!
%  2. Observables, if observ. are event-like then we can do it a history variable (as in KeY-ABS) otherwise we need symbolic-traces
% 3. For e.g. the array coordinate problem we need relational invariants in addition to relational postconditions (see mathias paper?)
% 


\begin{document}

\maketitle

\begin{abstract}
Relational verification through dynamic logic is a promising approach for verification of object oriented programs.
Recent advances from symbolic to abstract executions have enabled reasoning about incomplete/abstract versions of such programs.
This has proven fruitful in the exploration of correctness of refactorings primarily related to code blocks in Java.
In this paper we explore further types of equivalent transformations and refactorings and discuss the challenges that still
need to be overcome for full round-trip correctness of refactorings in object-oriented languages.
\end{abstract}

\section{Introduction}

Relational verification through dynamic logic is a promising approach for verification of object oriented programs.
Recent advances from symbolic to abstract executions have enabled reasoning about incomplete/abstract versions of such programs.
This has proven fruitful in the exploration of correctness of refactorings primarily related to code blocks in Java.
In this paper we explore further types of equivalent transformations and refactorings and discuss the challenges that still need to be overcome for full round-trip correctness of refactorings in object-oriented languages.

This paper is structured as follows: we first provide an overview on abstract executions, the main vehicle for correctness- and equivalence proofs.
Next we discuss issues related to formally correct and constructive handling of object creation to be able to express useful notions of equivalence in Section \ref{sec:objectcreation}.

Section \ref{sec:trace} moves away from only heap-related equivalence and discusses tracking equivalent observable behaviour in a relational framework.
The final challenge are relational invariants that we discuss in Section \ref{sec:relinv} which are required to track equivalent, yet structurally different representation of data structures.

%%% Local Variables:
%%% mode: latex
%%% TeX-master: "main"
%%% End:


\section{Preliminaries: Abstract Execution}\label{sec:prelim}
\begin{itemize}
\item \vsnote*{}{Ole}
\item difference SE/AE
\item AE constructs (statements, expression, blocks, constraints)
\item short subsection on how R. works (lhs/rhs setup, generating PO
  for KeY, KeY proves mostly automatic).
\end{itemize}


Let us consider the Extract Local Variable refactoring which has had dynamic detection considered \ref{stolz:isolarefa} for one of its pitfalls.

TODO: Add motivating java code & dynamic fix

\begin{figure}
  \centering
  \begin{subfigure}{.4\linewidth}
    \lstinputlisting[style=refinity]{ExtractVariable/REFINITY/before.refinity}
    \caption{Before}
  \end{subfigure}\hspace{1cm}
  \begin{subfigure}{.4\linewidth}
    \lstinputlisting[style=refinity]{ExtractVariable/REFINITY/after.refinity}
    \caption{After}
  \end{subfigure}
\captionof{lstlisting}{Extract Local Variable}
\label{lst:ExtractVariable-refinity}
\end{figure}


\section{Challenges}\label{sec:challenges}
In this section we explore the possibilities of \Refinity{} beyond the original work.
In particular, we are interested in moving away from statement-based refactorings to more complex changes that also affect the structure of the code.
The first new refactoring, Hide Delegate, harnesses Java inner classes to express the desired behaviour of \Refinity{} in a straight-forward manner
and KeY automatically completes the proof once the right preconditions are identified.

After that, we present a refactoring that is related to Dead Code Elimination.
This requires introduction of additional constructs for object-construction within KeY,
which are then harnessed in tactlets that express that certain non-identical heaps guarantee equivalent behaviour.

We continue our wishlist for further flexibility in expressing equivalent program behaviour based on the execution history of user-defined observable actions,
and finally discuss challenges related to expressing equivalence in the face of different data types.

\subsection{Encoding the Hide Delegate refactoring}\label{sec:hideDelegate}

The Hide Delegate refactoring can be described as an Extract Method refactoring on a call chain such as \jcode{Y y = o.f().g()}
the call chain is extracted to a method on \jcode{o} such that we can replace the chain with a call to the new method \jcode{o.h()}
which we declare as \lstinline|Y h() { return this.f().g(); }|.
\vsnote{I fixed a ``broken'' macro issue here.}
The refactoring can enable less
coupling as the class that contained the call chain afterwards does not need to know the return type of \jcode{f()}.
We note that this is just one example and intermediate, dotted, expressions can be arbitrary long and/or broken up across assignments to local variables.
\vsnote{Q: Generalize to \jcode{o = AE; z = o.g(); return z;} ?}
\vsnote{Three things on the nature of encoding things in R.: i) this is an example that R. is bad at \textit{names}: we need concrete names \texttt{f/g} (with concrete parameters), but want placeholders. ii) we don't really remove/replace anything, since we need all the code that we want to use in the same class, i.e., the ``new'' method \jcode{hideDelegate()} was already present before -- but that's okay because it's obviously correct to introduce methods that are not called -- another example of syntactical requirements on encodings in R. iii) This is the example for our abuse of R.\ for OO-based refactorings.}

\ref{lst:HideDelegate-nofields-refinity}


\begin{figure}
  \centering
  \begin{subfigure}[b]{.4\linewidth}
    \lstinputlisting[style=refinity]{HideDelegate/REFINITY/before.refinity}
    \caption{Before}
    \label{lst:HideDelegate-nofields-before-refinity}
  \end{subfigure}\hspace{1cm}
  \begin{subfigure}[b]{.4\linewidth}
    \lstinputlisting[style=refinity]{HideDelegate/REFINITY/after.refinity}
    \caption{After}
    \label{lst:HideDelegate-nofields-after-refinity}
  \end{subfigure}
\captionof{lstlisting}{Hide Delegate}
\label{lst:HideDelegate-nofields-refinity}
\end{figure}

\begin{figure}
  \centering
  \begin{subfigure}[b]{.4\linewidth}
    \lstinputlisting[style=refinity]{HideDelegate/REFINITY/Resource.refinity}
    \caption{Before}
    \label{lst:HideDelegate-nofields-resource-refinity}
  \end{subfigure}\hspace{1cm}
  \begin{subfigure}[b]{.4\linewidth}
    \lstinputlisting[style=refinity]{HideDelegate/REFINITY/Owner.refinity}
    \caption{After}
    \label{lst:HideDelegate-nofields-owner-refinity}
  \end{subfigure}
\captionof{lstlisting}{Hide Delegate Classes}
\label{lst:HideDelegate-nofields-classes-refinity}
\end{figure}

\subsection*{Hide Delegate --- once more with fields}

Similarily a minor variation of the  Hide Delegate refactoring also applies to a call chain which assigns to temporaries or fields
e.g. \lstinline[style=smallJava]|x = o.f(); Y y = x.g()|.
\vsnote*{Probably not true.}{%
Here we must beware that in our replacement we may observe a difference
if \lstinline[style=smallJava]|g()| can leave \lstinline[style=smallJava]|x| in an altered state which is then eliminated by applying
the refactoring \lstinline[style=smallJava]|Y y = h();|
}



%TODO this is not yet done

%%% Local Variables:
%%% mode: latex
%%% eval: (auto-fill-mode -1)
%%% TeX-master: "../main"
%%% End:

\subsection{Object creation}\label{sec:objectcreation}
\newcommand\relevant{$\mathrm{relevant}$}
\newcommand\assignable{$\mathrm{assignable}$}
\newcommand\accessible{$\mathrm{accessible}$}
\newcommand\keyrule[1]{\ensuremath{\mathrm{#1}}}

\begin{figure}
  \centering
  \begin{subfigure}{.2\linewidth}
    \lstinputlisting[style=refinity]{ObjectCreation/REFINITY/before.refinity}
    \caption{Before}
  \end{subfigure}\hspace{1cm}
  \begin{subfigure}{.2\linewidth}
    \lstinputlisting[style=refinity]{ObjectCreation/REFINITY/after.refinity}
    \caption{After}
  \end{subfigure}
\captionof{lstlisting}{Object creation}
\label{lst:ObjectCreation-refinity}
\end{figure}

As we have seen in the previous section, REFINITY encodes a rather harsh regimen on program equivalence:
in absence of a more fine-grained (application-specific) post-condition, it encodes that return values or exceptions must be identical on both sides,
as must be the objects in the \relevant{} location set (and the observables in this location set must be adequately specified).

This already creates the first hurdle for expressing useful notions of equivalences: as heap updates are symbolically encoded in the proofs, equivalent object-allocations are not directly identified by KeY as such.
Consider the following two pieces of code in listing (lis.)~\ref{lst:ObjectCreation-refinity}
if the two allocations are independent from each other, i.e. there is no dependency in their constructors, such as a shared global counter,
both programs are equivalent, yet cannot be proved automatically due to syntactically different heaps.
To address this limitation, we need additional rules that allow ignoring heap updates that are irrelevant.
Surprisingly, the extant literature does not provide much automation there, but rather builds on restrictive specifications wrt.\ \assignable{}/\accessible{} frames.

Coming back to our example, it is syntactically obvious that both sides yield different heaps.
Before addressing the underlying problem, \vsnote{@Eduard: Let's skip the obvious that KeY didn't know that two objects allocated in identical heaps are hence identical, or?}
we can immediately contribute a small taclet that states that two objects are identical, if their arguments to the constructor are equal and if any heap updates between the two allocations do not afffect the allocation of the second object.
Similar to KeY's \keyrule{dropUpdate_2}, we can easily eliminate these operations on the heap on unrelated \textit{types} and obtain equivalent programs.
\vsnote*{TODO: Eduard to elaborate a bit here.}{Unrelated types are of course only a shortcut, dropUpdate2 is much more specific, VS thinks -- but I can't decode the side conditions on the rule}.

Let us close with the remark that due to the nature of taclets, we have not so much proven this refactoring to be correct,
but rather (only) moved this decision further down the chain: unlike in a formalization of an OO model from ground up e.g.\ in the Coq-theorem prover,
we do not have a way of proving the taclets correct (i.e.\ derive them as lemmas) within KeY for \textit{any} program.
\vsnote{Eduard, pease confirm this claim and maybe adjust formulation to what's palpatable to our audience.}

\subsection*{Related: Dead Code Elimination}
Similar considerations of equivalent heap manipulations need to be considered also in the are of optimizations.
Take for example the trivial fragment in lis.~\ref{lst:xisnewxisnew}.
\begin{figure}
\centering
\begin{minipage}{.2\linewidth}
\begin{lstlisting}[style=refinity]
C x;
x = new C();
x = new C();
\end{lstlisting}
\end{minipage}
\captionof{lstlisting}{Dead Object}
\label{lst:xisnewxisnew}
\end{figure}
Again, without any sideeffects upon constructor invocation, the first update is of no consequence.
The additional rules above however will not yet be sufficient to prove that this version is equivalent to the version without the redundant object creation and assignment.
The location-set mechanism would still insist that the second object created  in the redundant version is a different object from the first (and only) object created in the optimized version.
On the one hand this can be addressed through a relaxed post-condition where we accept that we only need \textit{an} object of the right type and arguments, but it relies on the side-condition of the constructor not having side-effects, which we \vsnote*{TODO Eduard: is it?}{cannot easily encode as part of the post-condition}.
We face the same issue if we would want to encode the absence of side-effects as precondition to a taclet.

%%% Local Variables:
%%% mode: latex
%%% TeX-master: "main"
%%% End:


\subsection{Trace properties}\label{sec:traces}
In the previous section, we have established that syntactical equality of heaps is too strict as a precondition to equivalent (or rather: identical) behaviour.
In practice, developers are content if refactored code gives the same observable behaviour \cite{needed}.
This behaviour is first and foremost encoded through unit-tests, but also on tests on output (e.g.\ via \texttt{print}-statements).
Here we then have a much more relaxed setting where equivalence is decoupled from the fine-grained program semantics.

Harnessing output as a mechanism to decide equivalence requires instrumentation of both software versions, though:
we need to extend our code with a (global) variable that accumulates observable behaviour at dedicated locations in the program.
Making this value the return value on both sides allows us to leverage the colismponent-wise equality in the generic post-condition that REFINITY generates,
and achieves the desired effect when we completely ignore differences in the relevant locations set by not including any variables in there at all.

We note that this places more burden on the developer, as they have to make sure that their chosen encoding of events in the data structure captures the salient part of equivalence.
Clearly, this data structure is sensitive to order and any object identities very much in the same way as is the syntactic representation of heap manipulations.
This explicit encoding of user-defined execution histories has the advantage that it is not only useful for proofs, but can also directly be harnessed in concrete unit-tests.
\vsnote{Just a minor observation.}

\vsnote*{}{TODO: Any good example here?}

\vsnote*{}{TODO: Idea can't be new. Also see histories in general (Eduard?)}

%%% Local Variables:
%%% mode: latex
%%% TeX-master: "main"
%%% End:


\subsection{Relational invariants}\label{sec:relational}
Another area of interest for equivalence is replacing one data structure with another,
e.g.\ Fowler's \rname{Replace Array with Object}~\cite[p.186]{fowler:refactoring} or \rname{Replace Primitive with Object}~\cite{fowler:refactoring2nd}.
As an example, in the following we look at a piece of code where an array is replaced with an object (or vice versa).
Again, from the strict default perspective of ``equal return values, equal heaps'', any two programs using the data structures are obviously not equal.
Encoding observability through traces as per the previous section will obviously solve this issue.
A new challenge arises when both programs use different or disjoint sets of operations, i.e.\ we have different alphabets for their trace languages.

Consider a refactoring that replaces an array containing position given by a latitude and longitude as in fig.~\ref{refa:ReplaceArray-before} with an object that gives read or write access to the same values through setters and getters that make it immediately clear what is being accessed as seen in fig.~\ref{refa:ReplaceArray-after}.

For this to be a refactoring we must be certain that indexing can only occur in bounds for the original program as there will be no corresponding out of bounds failure for method calls. There may be usage sites in the original program which in some way rely on the array representation, these must be identified and handled in an equivalent way in the refactored program.

\begin{figure}
  \begin{subfigure}[h]{.45\linewidth} 
    \lstinputlisting[style=smallJava]{ReplaceArrayWithObject/Java/before.java}
    \caption{Before}
    \label{refa:ReplaceArray-before}   
  \end{subfigure}\hspace{1cm}
  \begin{subfigure}[h]{.45\linewidth}
    \lstinputlisting[style=smallJava]{ReplaceArrayWithObject/Java/after.java}
    \caption{After}
    \label{refa:ReplaceArray-after}
  \end{subfigure}
  \caption{Replace array with object refactoring}
  \label{refa:ReplaceArray}
\end{figure}

In REFINITY it would be difficult to even model this because what we would most naturally want to do is to somehow write a LHS with an abstract statement $P$ and a RHS with an abstract statemetn $Q$ and specify that $P$ contains an array used to index some data as seen in fig.~\ref{refa:ReplaceArray}. Then we would want to require somehow that $Q$ is exactly as $P$ but with the array substituted for a data object and any reads or writes to the array replaced with appropriate setters and getters. Such things are not possible as on the modeling level the abstract elements operate as black boxes we can't whose contents we can't specify. 



%%% Local Variables:
%%% mode: latex
%%% eval: (auto-fill-mode -1)
%%% TeX-master: "main"
%%% End:


\section{Discussion}\label{sec:discussion}
\begin{itemize}
\item Challenge: be careful in encoding, since currently LHS and RHS share classes and methods.
\item Adressing this would essentially require choosing unique names on LHS and RHS, since they go into a single KeY-proof, and hence would require some form of mapping objects between both sides.
\item Challenge: capturing the most general instance of a pattern (example: Extract Local Variable, with the duplicate call to \texttt{n()}).
\item For example for HD, we need to have two cases, one with fields, in addition to the one with local vars.
\item We feel that often pre-conditions on a refactoring can easily be discharged by simple syntactical or static analysis (e.g. ``code does never read attribute $x$ of objects of type $C$'').
Yet unlike in other formal work where the program is encoded as part of the proof-term, we cannot implement such analyses within KeY, and can only informally document and require such side-conditions on refactorings.
\end{itemize}


%%% Local Variables:
%%% mode: latex
%%% eval: (auto-fill-mode -1)
%%% TeX-master: "main"
%%% End:


\section{Related Work}\label{sec:related}

\begin{itemize}
\item formal proofs of correctness for OO program optimisation
\item formal proofs of OO refactoring
\item check Dominic's thesis and papers for the relevant related work
\item important: correctness on actual Java, not something which looks like Java
\item also: correctness with respect to OO features
\end{itemize}

Similar or other approaches to formal verification of refactorings can be found in work by
Garrido et. al. \cite{garrido2006formal} who formalize \emph{``Push Down Method''},
\emph{``Pull Up Field''} and \emph{``Rename Temporary''} using an executable
Java formal semantics in Maude and give partially mechanized proofs for the two former.

Long Quan et. al. \cite{DBLP:conf/isola/QuanQL08} formulate refactorings as refinement
laws in the calculus of refinement of component and object-oriented systems (rCOS),
focusing on correctnes proofs of refactoring rules themselves.

While KeY and REFINITY is unique for its relational verification capacity for
schematic programs (or abstract programs) it is limited in power for verification
of concrete programs relying much more on manual specification or interaction \cite{DBLP:conf/aplas/Steinhofel20}
than tools like LLRêve \cite{DBLP:journals/jar/KieferKU18} or SymDiff \cite{DBLP:conf/cav/LahiriHKR12}
which offer more automation for concrete programs.





\section{Conclusion}\label{sec:conclusion}

We have presented two new encodings of refactorings (\rname{Extract Local Variable} and \rname{Hide Delegate}) and their necessary preconditions (constraints) for them to be behaviour-preserving for \Refinity{}.
\Refinity{} has syntactical constructs that capture abstract program executions, for which the KeY system,
an automated theorem prover for JavaDL, succeeds in proving the refactorings as correct (equal) wrt.\ to the Java semantics without user interaction.

The \rname{Hide Delegate} refactoring departs from statement-based refactorings and considers changes involving multiple classes and requires us to consider the first subtle difference between equivalent-yet-not-equal objects in the form of equivalent exceptions and how we need to explicitly address this in the post-condition of the proof-obligation.
This also allows us to make a contribution through further taclets that capture some indistinguishable programs that only differ in placement of objects on the heap.

We discuss in how far \Refinity{} could be used to capture other refactorings, broadening our investigation into the underlying notion of sometimes use-case specific \textit{equivalent behaviour}.
For example, while traces over observable behaviour could be explicitly encoded and checked against each other as return values in \Refinity{}, a more general process-algebra inspired approach of execution histories for abstract executions would avoid confounding the original program logic with scaffolding to achieve an encoding without having to extend the tool.
We also point out the difficulties due to a naming issue in the current encoding from Refinity to KeY proof-obligations for refactorings that change the class hierarchies or in general attempt to relate behaviour across different types.

\paragraph{Future Work.} 
To investigate the discussed problems with refactoring using AE wrt.\ trace properties, we are currently implementing AE for BPL in the Crowbar tool~\cite{crowbar} as a starting point, a symbolic execution engine to prototype behavioral symbolic execution.

We are also working on contributing further encodings of common refactorings that can already now be handled by \Refinity{}.
In addition, we are particularly interested in additional taclets for KeY that would enable automation of proofs
that currently are stuck on the explicit symbolic encoding of program state although it would be indistinguishable from equivalent states in pratice.
The latter is of relevance e.g. for proving common optimizations as in our \rname{Dead Code Elimination} example.

\subsubsection*{Acknowledgements}
This work was partially supported by the Research Council of Norway via \texttt{SIRIUS} (237898), \texttt{PeTWIN} (294600) and \texttt{CROFLOW} (326249).
%\input{conclusion}

% \section{Leftovers}
% \textit{Use if space/time/useful.}
% \subsection{Remove assignment to parameter}
% \begin{figure}
  \begin{subfigure}[h]{.45\linewidth}
    \lstinputlisting[style=smallJava]{RemoveAssignmentToParameter/Java/before.java}
    \caption{Before}
    \label{refa:RemoveAssignment-before}    
  \end{subfigure}\hspace{1cm}
  \begin{subfigure}[h]{.45\linewidth}
    \lstinputlisting[style=smallJava]{RemoveAssignmentToParameter/Java/after.java}
    \caption{After}
    \label{refa:RemoveAssignment-after}
  \end{subfigure}
  \caption{Replace array with object refactoring}
  \label{refa:RemovAssignment}
\end{figure}


\bibliographystyle{splncs04}
\bibliography{refs}

\end{document}

%%% Local Variables:
%%% mode: latex
%%% eval: (auto-fill-mode -1)
%%% TeX-master: t
%%% End:
