
Relational verification through dynamic logic is a promising approach for verification of refactorings in object oriented programs.
Recent advances from symbolic to Abstract Execution (AE)~\cite{DBLP:conf/fm/SteinhofelH19} have enabled reasoning about incomplete/abstract versions of such programs.
This has proven fruitful in the exploration of correctness of refactorings primarily related to code blocks in Java:
%In this paper we explore further types of equivalent transformations and refactorings and discuss the challenges that still need to be overcome for full round-trip correctness of refactorings in object-oriented languages.
%
AE introduces abstract statements (and expressions), which act as named and specified placeholders for some statement of the host language.
A refactoring proof is a relational verification (RV) proof that compares two programs, which have the \emph{same} abstract statements. One of the programs represents the schema of the code before the refactoring, and the other one the code afterwards, with respect to some relational post-condition that defines the notion of \emph{program equivalence}.

\paragraph{Beyond Statement-Based Refactorings.}
Abstract statements have been used to verify the correctness of refactorings that are based on moving code within a method and on extracting some statements into an own method (the \texttt{Extract Method} refactoring). 
\eknote*{@VS please extend and provide a reference.}{However, refactorings a rarely confined into single methods~\cite{???}, but also require to restructure classes and data structures.}


Here, we investigate more complex refactorings that combine code movement beyond multiple methods. 
\eknote*{@VS please extend }{For one, we investigate the \texttt{Hide Delegate} refactoring.}


\paragraph{Challenges.}
More complex refactorings need more elaborate specification and verification techniques for RV. 
The reason is that the programs surrounding the abstract statements, as well as the notion of equivalence becomes more involved.
For example, if the surrounding programs throw exceptions, under which conditions are the exceptions considered equivalent? 
As the exceptions are just objects, there are possible choices when newly created objects are considered equal in RV using dynamic logic, 
first investigated by Beckert et al.~\cite{DBLP:conf/lopstr/BeckertBKSSU13}.



For exceptions and object creations, we describe several possibilities when newly created objects (resp.\ thrown exceptions) are considered equal
and how this information can be used in dynamic logic proofs. The different possibilities are implemented as different rule sets and must not be encoded
in the relational post-condition. This reduces the size of the required specification, a notorious bottleneck in formal verification~\cite{DBLP:journals/corr/abs-1211-6186,DBLP:series/lncs/HahnleH19}.


Furthermore, we discuss the necessary extensions needed to prove the equivalence where one data strucute is replaced by another, e.g., an array by a special record.
Here, the main challenge lies in the encoding that the structures are used correctly \emph{throughout} the execution. This could, for example, be handled by coupled invariants~\cite{DBLP:conf/birthday/BeckertU18}, whose connection to AE is unclear. 
Lastly, we discuss the challenge to apply AE to novel specification approaches for \emph{traces} which aim to simplify the specification of temporal properties for expressive properties, but whose use for RV is unexplored.

\paragraph{Contributions and Structure.}
Our contributions are (1) a proof of the correctness of the hide-delegate refactoring, that uses a refactoring beyond the move of code within a method, highlights the interaction of AE with general RV challenges, as well as (2) a discussion of possible extensions that would be required for the use of abstract execution for further refactorings.

%This paper is structured as follows. 
We first describe AE and RV using the \texttt{Extract Local Variable} refactoring in Sec.~\ref{sec:prelim}, before we discuss the challenges for AE in Sec.~\ref{sec:challenges}. Sec.~\ref{sec:hide} describes and proves the \texttt{Hide Delegate} refactoring(s). Sec.~\ref{sec:discussion} discusses our results and Sec.~\ref{sec:related} gives the related work. Lastly, Sec.~\ref{sec:conclusion} concludes.



%This paper is structured as follows: we first provide an overview on abstract executions, the main vehicle for correctness- and equivalence proofs, and their use in REFINITY.
%Next, we provide an encoding of the Hide Delegating refactoring for REFINITY that moves away from statement-based refactorings to object-oriented refactorings in Section \ref{sec:hideDelegate}.
%After that in Section \ref{sec:challenges}, we explore (and address, where possible) \vsnote*{Not really limitations...better word?}{limitations} in REFINITY that we have discovered by investigating further refactorings.
%Our main goal is to extend the notion of \textit{equal} programs to \textit{equivalent} programs, e.g. by replacing data types.
%\vsnote{Mention possible application to optimizations already in intro? Probably.}
%We conclude with a discussion in Section \ref{sec:discussion} on the feasibility of necessary developments \vsnote*{esp. if we want to use histories in subsequent proofs.}{and their dependencies}.
% VS: This will then be included in the section above:
% Next we discuss issues related to formally correct and constructive handling of object creation to be able to express useful notions of equivalence in Section \ref{sec:objectcreation}.
% Section \ref{sec:trace} moves away from only heap-related equivalence and discusses tracking equivalent observable behaviour in a relational framework.
% The final challenge are relational invariants that we discuss in Section \ref{sec:relinv} which are required to track equivalent, yet structurally different representation of data types.

%%% Local Variables:
%%% mode: latex
%%% TeX-master: "main"
%%% End:
