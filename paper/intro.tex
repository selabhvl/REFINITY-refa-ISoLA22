
Relational verification through dynamic logic is a promising approach for verification of object oriented programs.
Recent advances from symbolic to abstract executions have enabled reasoning about incomplete/abstract versions of such programs.
This has proven fruitful in the exploration of correctness of refactorings primarily related to code blocks in Java.
In this paper we explore further types of equivalent transformations and refactorings and discuss the challenges that still need to be overcome for full round-trip correctness of refactorings in object-oriented languages.

This paper is structured as follows: we first provide an overview on abstract executions, the main vehicle for correctness- and equivalence proofs.
Next we discuss issues related to formally correct and constructive handling of object creation to be able to express useful notions of equivalence in Section \ref{sec:objectcreation}.

Section \ref{sec:trace} moves away from only heap-related equivalence and discusses tracking equivalent observable behaviour in a relational framework.
The final challenge are relational invariants that we discuss in Section \ref{sec:relinv} which are required to track equivalent, yet structurally different representation of data structures.

%%% Local Variables:
%%% mode: latex
%%% TeX-master: "main"
%%% End:
