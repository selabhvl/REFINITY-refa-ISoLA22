Let us consider the \rname{Extract Local Variable} refactoring seen in Listing~\ref{lst:ExtractVariable-java}.
The example is an instance of a more general case, where preserving the behaviour of the program depends on other parts of the code.
The behaviour of the program changes if the method call \jcode{n()} has access to the attribute \jcode{x} and overwrites it.

Before we would have potentially method calls to different objects \jmcode{o$_1$.n()} and then \jmcode{o$_2$.n()}, whereas after applying the refactoring we would
have \jmcode{o$_1$.n()} followed by \jmcode{o$_1$.n()}.
In this case if, e.g., \jcode{n()} simply prints \jcode{this.toString()}, we will observe a difference in the two programs.

\begin{figure}[tbp]
  \captionsetup{type=lstlisting}
  \centering
  \begin{sublstlisting}[b]{.2\linewidth}
    \lstinputlisting[style=refinity]{ExtractVariable/Java/before.java}
    \caption{Before}
  \end{sublstlisting}\hspace{1.5cm}
  \begin{sublstlisting}[b]{.34\linewidth}
    \lstinputlisting[style=refinity]{ExtractVariable/Java/after.java}
    \caption{After}
  \end{sublstlisting}
  \caption{\rname{Extract Local Variable} refactoring}
  \label{lst:ExtractVariable-java}
\end{figure}

The dynamic check for such a change detailed in other work~\cite{stolz:isolarefa} codified the necessity that the reference \jcode{x}
remains unchanged through the introduction of an assertion \jcode{assert temp == x;} to uncover violations after the fact,
which is useful, e.g., when checking the refactored code against its suite of unit tests.

In REFINITY one proceeds to verify a refactoring as correct by supplying the code before and after refactoring, and supplying a desired precondition and postcondition to relate the two programs.
We essentially ask REFINITY: \emph{given these preconditions does the postcondition hold after abstract execution of these two abstract programs?}
%REFINITY accepts a restricted
%subset of Java, the same restrictions present in KeY, with extensions for abstract program fragments; pieces that are ``placeholders'' for concrete Java code.
A proof of correctness in REFINITY is a proof for any concrete Java programs that can be \emph{instantiated} to adhere to the abstract specification given in REFINITY.

\begin{figure}[tbp]
  \captionsetup{type=lstlisting}
  \centering
  \begin{sublstlisting}[b]{.34\linewidth}
    \lstinputlisting[linerange={2-5},style=refinity]{ExtractVariable/REFINITY/before.refinity}
    \vspace{-2mm}
    \caption{Before}
    \label{lst:ExtractVariable-refinity-before}
  \end{sublstlisting}\hspace{1cm}
  \begin{sublstlisting}[b]{.34\linewidth}
    \lstinputlisting[linerange={2-6},style=refinity]{ExtractVariable/REFINITY/after.refinity}
    \vspace{-2mm}
    \caption{After}
    \label{lst:ExtractVariable-refinity-after}
  \end{sublstlisting}
\caption{\rname{Extract Local Variable} refactoring in REFINITY}
\label{lst:ExtractVariable-refinity}
\end{figure}

We describe the refactoring to REFINITY as shown with a left side (Before) in Listing~\ref{lst:ExtractVariable-refinity-before},
a right side (After) in Listing~\ref{lst:ExtractVariable-refinity-after} and a method level context which contains the method we have already shown in Listing~\ref{lst:ExtractVariable-refinity-method}.

Additionally, some information is declared in parts of REFINITY's interface that are not shown here:
Free program variables, here \rcode{x}; abstract location sets, here \rcode{frN} and \rcode{fpN}; relevant locations for the before and after code, here empty;
the desired precondition, here empty, and the desired postcondition.

The pre- and postcondition are specified in terms of equations that may relate to the effects of the before and after side, such as return values, exceptions thrown and any of the relevant locations declared.
\OA{Here we use the default postcondition REFINITY provides which is simply \rcode{\\result\_1 == \\result\_2}, where \rcode{\\result\_1} and \rcode{\\result\_2}
are each respectively a sequence of results for the abstract execution of the Before and After program.
Each sequence contain in this order:
1) The return value if any, otherwise null;
2) exceptions thrown if any, otherwise null;
and 3) the values at specified abstract location sets (that the user selects as relevant).}

%\vsnote*{Unclear (yes, I know what is meant). More constructive: ``an assertion that the values for all locations as per the relevant set are identical''?}{the values at relevant abstract location sets}.
\OA{The default postcondition in our case will be that \rcode{x} returned in Before must be identical to \rcode{temp} returned in After, and that any exception thrown must be identical for the
Before and After program. When left empty, as done here and so
trivially true, equality with respect to the last component of the result sequences is an assertion that the values at all abstract location sets selected as relevant are identical.}

We use an assertion to ensure that we consider only the refactoring when~\rcode{x} is an instance of its intended type.
A current limitation of the implementation of REFINITY necessitate the casting \rcode{(X)x} as free program variables may only be declared to be of type Object.

With only the specification shown in Listing~\ref{lst:ExtractVariable-refinity}, REFINITY is unable to prove that \rcode{x} will be identical to \rcode{temp} after AE of both sides.
We get several instances of SE resulting in the reference \rcode{x} being changed by the abstract statement~\rcode{N} in~\rcode{n()}.

To prohibit this, we specify that the execution of the abstract statement \rcode{N} cannot interfere with \rcode{x}.
A constraint on the frame \rcode{frN} of abstract statement \rcode{N},
namely that it is disjoint from \rcode{x} must be introduced, which is
achieved by putting an abstract execution constraint \rcode{@ ae\_constraint \\disjoint(x,frN)} on each side which ensures \rcode{x}
will be assumed not to be in \rcode{frN}.

After the aforementioned change, REFINITY manages to automatically prove that the postcondition holds after abstract execution of both sides;
\rcode{x} and \rcode{temp} will hold identical references and any exceptions thrown will be identical.
\vsnote{New, please check.}{Note that it remains for any concrete application of this refactoring to prove that any code matching abstract statements fulfils their annotated constraints.}

%\vsnote*{Unclear. Why ``partial''?}{One can consider what we have managed to encode and prove a partial success.}
%\vsnote{Unclear/can't follow.}{Yet this is a slightly unnatural encoding if we are to consider concrete instantiations of what we have expressed here.
\OA{Although we successfully prove the refactoring we have encoded here, the required scaffolding of return statements leaves something to be desired:
Finding intended concrete instances of the abstract programs now involve, potentially, ignoring the return statements.}
%%% Local Variables:
%%% mode: latex
%%% TeX-master: "../main"
%%% End:
