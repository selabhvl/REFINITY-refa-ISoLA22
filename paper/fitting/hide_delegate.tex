\subsection{Encoding the Hide Delegate refactoring}\label{sec:hideDelegate}

The \rname{Hide Delegate} refactoring can be described as an \rname{Extract Method} refactoring on a call chain such as seen in \jcode{Y y = o.f().g()}
where the call chain is extracted to a new method on \jcode{o}, say \jcode{h()} which contains the extraction \jcode{Y h() \{ return this.f().g(); \}}, such that we replace the chain as seen in \jcode{Y y = o.h()}.

\begin{figure}[tbp]
  \centering
  \begin{subfigure}{.3\linewidth}
    \begin{tikzpicture}[node distance=1cm,auto,>=stealth']
    \node[] (this) {\jcode{this}};
    \node[right = of this] (o) {\jcode{o}};
    \node[right = of o] (null) {\jcode{null}};
    
    \node[below of=this, node distance=3cm] (thisb) {};
    \node[below of=o, node distance=3cm] (ob) {};
    \node[below of=null, node distance=3cm] (nullb) {};

    \draw (this) -- (thisb);
    \draw (o) -- (ob);
    \draw (null) -- (nullb);

    \draw[->] ($(this)!0.25!(thisb)$) -- node[above,scale=1,pos=0.2]{\jcode{f()}} ($(o)!0.25!(ob)$);
    \draw[->] ($(this)!0.45!(thisb)$) -- node[above,scale=1,pos=0.1]{\jcode{g()}} node[below,scale=1,near start]{NPE} ($(null)!0.45!(nullb)$);
\end{tikzpicture}

    \caption{Before}
    \label{fig:hd-npe-before}
  \end{subfigure}
  \begin{subfigure}{.3\linewidth}
    \begin{tikzpicture}[node distance=1cm,auto,>=stealth']
    \node[] (this) {\jcode{this}};
    \node[right = of this] (o) {\jcode{o}};
    \node[right = of o] (null) {\jcode{null}};
    
    \node[below of=this, node distance=3cm] (thisb) {};
    \node[below of=o, node distance=3cm] (ob) {};
    \node[below of=null, node distance=3cm] (nullb) {};

    \draw (this) -- (thisb);
    \draw (o) -- (ob);
    \draw (null) -- (nullb);
    \coordinate (loop) at ($(o)!0.35!(ob)$);

    \draw[->] ($(this)!0.25!(thisb)$) -- node[above,scale=1,pos=0.2]{\jcode{h()}} ($(o)!0.25!(ob)$);
    \draw[->,rounded corners=2pt] (loop) -- ($(loop) + (0.5,0) $) --  ($(loop) + (0.5,-0.1) $)  node[midway,scale=1]{\jcode{f()}} -- ($(loop) + (0,-0.1)$);    
    \draw[->] ($(o)!0.75!(ob)$) -- node[above,scale=1,pos=0.2]{\jcode{g()}} node[below,scale=1,midway]{NPE} ($(null)!0.75!(nullb)$);
\end{tikzpicture}

    \caption{After}
    \label{fig:hd-npe-after}    
  \end{subfigure}
  \caption{NPEs}
  \label{fig:NPEs}
\end{figure}

We note that a pattern usually considered is also \jcode{X x = o.f(); Y y = x.g()} potentially with non-interfering intermediate statements inbetween the two statements,
but in that case we can reach the considered pattern through applications of \rname{Slide Statement} and finally \rname{Inline Variable} on \jcode{x}.
The refactoring can enable less coupling as the class that contained the call chain afterwards does not need to know the return type of \jcode{f()}.

Also of note is the scenario shown in Fig.~\ref{fig:NPEs} where the call to \jcode{f()} returns \jcode{null}. In the strictest sense of behavioral preservation
we will observe a difference as the \jcode{NullPointerException} (NPE) thrown before Fig.~\ref{fig:hd-npe-before} will show a different stacktrace than the one thrown after Fig.~\ref{fig:hd-npe-before}.
Thus we consider a behavioral equivalence that allows for disagreement in stacktraces for such matching exceptions; in fact one is unable to make any other distinction in \Refinity{}.

%\vsnote{Three things on the nature of encoding things in R.: i) this is an example that R. is bad at \textit{names}: we need concrete names \texttt{f/g} (with concrete parameters), but want placeholders. ii) we don't really remove/replace anything, since we need all the code that we want to use in the same class, i.e., the ``new'' method \jcode{hideDelegate()} was already present before -- but that's okay because it's obviously correct to introduce methods that are not called -- another example of syntactical requirements on encodings in R. iii) This is the example for our abuse of R.\ for OO-based refactorings.}
%\vsnote{TODO @Ole: check that in HD we explain the ``equivalence'' of the NPEs --- I don't see this yet in the example in 3.1}

\begin{figure}[tbp]
  \captionsetup{type=lstlisting}
  \centering
  \begin{sublstlisting}[b]{.455\linewidth}
    \lstinputlisting[style=refinity]{HideDelegate/REFINITY/before.refinity}
    \caption{Before}
    \label{lst:HideDelegate-nofields-before-refinity}
  \end{sublstlisting}\hspace{1cm}
  \begin{sublstlisting}[b]{.455\linewidth}
    \lstinputlisting[style=refinity]{HideDelegate/REFINITY/after.refinity}
    \caption{After}
    \label{lst:HideDelegate-nofields-after-refinity}
  \end{sublstlisting}
\caption{Hide Delegate in REFINITY}
\label{lst:HideDelegate-nofields-refinity}
\end{figure}

\begin{figure}[tbp]
  \captionsetup{type=lstlisting}
  \centering
  \begin{sublstlisting}[b]{.4\linewidth}
    \lstinputlisting[style=refinity]{HideDelegate/REFINITY/Resource.refinity}
    \caption{Before}
    \label{lst:HideDelegate-nofields-resource-refinity}
  \end{sublstlisting}\hspace{1cm}
  \begin{sublstlisting}[b]{.4\linewidth}
    \lstinputlisting[style=refinity]{HideDelegate/REFINITY/Owner.refinity}
    \caption{After}
    \label{lst:HideDelegate-nofields-owner-refinity}
  \end{sublstlisting}
\caption{Hide Delegate Classes in REFINITY}
\label{lst:HideDelegate-nofields-classes-refinity}
\end{figure}
We specify the Before and After program fragment for a \rname{Hide Delegate} refactoring in Listing~\ref{lst:HideDelegate-nofields-refinity} which contains no surprises when compared to the sketched out example above.
The classes and methods used in the refactoring appear in Listing~\ref{lst:HideDelegate-nofields-classes-refinity} and show that we minimally specify the contents of the involved methods by using abstract
statements in their bodies.
Note that we do not require that the abstract statements \rcode{F} and \rcode{G} in the methods \rcode{getOwner()} and \rcode{getResource} cannot complete abruptly.
That means the abstract statements may for instance throw exceptions; for instance the sketched out scenario considered in Fig.~\ref{fig:NPEs} where here \rcode{getResource()} will return \rcode{null} and cause the following call to throw a NPE is a possibility.

In order to prove the specified \rname{Hide Delegate} refactoring in the published version of \Refinity{} one must write a postcondition that consists of a conjunction of return values of the sides being identical and
that exceptions thrown are both instances of NPE or equal.
This is owing to the fact that \Refinity{} does not consider occurances of \jcode{new NullPointerException()}, or any other newly created objects, to be equal.

In our modified version of \Refinity{} this is no longer an issue and we may use the default postcondition that simply matches return results and exceptions;
\Refinity{} will automatically manage prove the shown \rname{Hide Delegate} refactoring to be correct wrt. the given postconditions.


%%% Local Variables:
%%% mode: latex
%%% eval: (auto-fill-mode -1)
%%% TeX-master: "../main"
%%% End:
