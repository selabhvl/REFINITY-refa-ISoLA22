\subsection{Encoding the Hide Delegate refactoring}\label{sec:hideDelegate}

The \rname{Hide Delegate} refactoring can be described as an \rname{Extract Method} refactoring on a call chain such as seen in \jcode{Y y = o.f().g()}
where the call chain is extracted to a new method on \jcode{o}, say \jcode{h()}, which contains the extraction \jcode{Y h() \{ return this.f().g(); \}}, such that we can replace the chain above with \jcode{Y y = o.h()}.
The refactoring can enable less coupling as the class that contained the call chain afterwards does not need to know the return type of \jcode{f()}.

\begin{figure}[tbp]
  \centering
  \begin{subfigure}{.3\linewidth}
    \begin{tikzpicture}[node distance=1cm,auto,>=stealth']
    \node[] (this) {\jcode{this}};
    \node[right = of this] (o) {\jcode{o}};
    \node[right = of o] (null) {\jcode{null}};
    
    \node[below of=this, node distance=3cm] (thisb) {};
    \node[below of=o, node distance=3cm] (ob) {};
    \node[below of=null, node distance=3cm] (nullb) {};

    \draw (this) -- (thisb);
    \draw (o) -- (ob);
    \draw (null) -- (nullb);

    \draw[->] ($(this)!0.25!(thisb)$) -- node[above,scale=1,pos=0.2]{\jcode{f()}} ($(o)!0.25!(ob)$);
    \draw[->] ($(this)!0.45!(thisb)$) -- node[above,scale=1,pos=0.1]{\jcode{g()}} node[below,scale=1,near start]{NPE} ($(null)!0.45!(nullb)$);
\end{tikzpicture}

    \caption{Before}
    \label{fig:hd-npe-before}
  \end{subfigure}
  \begin{subfigure}{.3\linewidth}
    \begin{tikzpicture}[node distance=1cm,auto,>=stealth']
    \node[] (this) {\jcode{this}};
    \node[right = of this] (o) {\jcode{o}};
    \node[right = of o] (null) {\jcode{null}};
    
    \node[below of=this, node distance=3cm] (thisb) {};
    \node[below of=o, node distance=3cm] (ob) {};
    \node[below of=null, node distance=3cm] (nullb) {};

    \draw (this) -- (thisb);
    \draw (o) -- (ob);
    \draw (null) -- (nullb);
    \coordinate (loop) at ($(o)!0.35!(ob)$);

    \draw[->] ($(this)!0.25!(thisb)$) -- node[above,scale=1,pos=0.2]{\jcode{h()}} ($(o)!0.25!(ob)$);
    \draw[->,rounded corners=2pt] (loop) -- ($(loop) + (0.5,0) $) --  ($(loop) + (0.5,-0.1) $)  node[midway,scale=1]{\jcode{f()}} -- ($(loop) + (0,-0.1)$);    
    \draw[->] ($(o)!0.75!(ob)$) -- node[above,scale=1,pos=0.2]{\jcode{g()}} node[below,scale=1,midway]{NPE} ($(null)!0.75!(nullb)$);
\end{tikzpicture}

    \caption{After}
    \label{fig:hd-npe-after}    
  \end{subfigure}
  \caption{NPEs}
  \label{fig:NPEs}
\end{figure}

We note that in the case of the more general pattern \jcode{X x = o.f(); Y y = x.g()} with non-interfering intermediate statements in between the two statements,
we can reach the considered pattern through applications of \rname{Slide Statement} and finally \rname{Inline Variable} on \jcode{x}.


The scenarios shown in the sequence diagrams in Fig.~\ref{fig:NPEs} will both result in \jcode{NullPointerException} (NPE) when the call to \jcode{f()} returns \jcode{null}.
In the strictest sense of behavioral preservation
we will observe a difference as the NPE thrown before refactoring (Fig.~\ref{fig:hd-npe-before}) will show a different stacktrace than the one thrown after refactoring (Fig.~\ref{fig:hd-npe-after}).
Thus we consider a behavioral equivalence that allows for disagreement in stacktraces for such matching exceptions; in fact one is unable to make any other distinction in \Refinity{}
as it does not consider such effects.
%\vsnote{Something about post-condition missing here? I don't remember, does R. not actually model the stacktrace in an exc. anyways or did we need something explicit in the post-cond?}
% Ole: post-condition mention comes below the figure below

\begin{figure}[tbp]
  \captionsetup{type=lstlisting}
  \centering
  \begin{sublstlisting}[b]{.455\linewidth}
    \lstinputlisting[style=refinity]{HideDelegate/REFINITY/before.refinity}
    \caption{Before}
    \label{lst:HideDelegate-nofields-before-refinity}
  \end{sublstlisting}\hspace{1cm}
  \begin{sublstlisting}[b]{.455\linewidth}
    \lstinputlisting[style=refinity]{HideDelegate/REFINITY/after.refinity}
    \caption{After}
    \label{lst:HideDelegate-nofields-after-refinity}
  \end{sublstlisting}
\caption{Hide Delegate in REFINITY}
\label{lst:HideDelegate-nofields-refinity}
\end{figure}

\begin{figure}[tbp]
  \captionsetup{type=lstlisting}
  \centering
  \begin{sublstlisting}[b]{.4\linewidth}
    \lstinputlisting[style=refinity]{HideDelegate/REFINITY/Resource.refinity}
    \caption{Before}
    \label{lst:HideDelegate-nofields-resource-refinity}
  \end{sublstlisting}\hspace{1cm}
  \begin{sublstlisting}[b]{.4\linewidth}
    \lstinputlisting[style=refinity]{HideDelegate/REFINITY/Owner.refinity}
    \caption{After}
    \label{lst:HideDelegate-nofields-owner-refinity}
  \end{sublstlisting}
\caption{Hide Delegate Classes in REFINITY}
\label{lst:HideDelegate-nofields-classes-refinity}
\end{figure}
We specify the Before and After program fragment for a \rname{Hide Delegate} refactoring in Listing~\ref{lst:HideDelegate-nofields-refinity} which faithfully captures the previously sketched out refactoring.
The classes and methods used in the refactoring appear in Listing~\ref{lst:HideDelegate-nofields-classes-refinity} and show that we minimally specify the contents of the involved methods by using abstract
statements in their bodies.
Note that we allow abrupt completion in the abstract statements \rcode{F} and \rcode{G} in the methods \rcode{getOwner()} and \rcode{getResource}.
That means the abstract statements may for instance throw exceptions; the sketched out scenario considered in Fig.~\ref{fig:NPEs} where \rcode{getResource()} will return \rcode{null} and cause the following call to throw a NPE is a possibility.

In order to prove the specified \rname{Hide Delegate} refactoring in the \vsnote*{Confusing, is this is ours or the historic version}{published} version (v0.9.7) of \Refinity{} we need a postcondition that consists of a conjunction of return values of the before- and after-programs being identical and
that any thrown exceptions are both instances of NPE or otherwise equal.
This is owing to the fact that \Refinity{} does not consider occurrences of \jcode{new NullPointerException()}, or any other newly created objects, to be equal.
\vsnote{Is object-creation the only issue; see ``stacktraces'' above? I can't remember. In any, case it's probably JavaDL/KeY, not R's ``fault''.}

\OA{We have improved  \Refinity{}\footnote{available at \tiny \url{https://github.com/selabhvl/REFINITY-abstractallocate}} such that this is no longer an issue; we may keep the default postcondition that simply matches return results and exceptions, and
\Refinity{} automatically manages to prove the shown \rname{Hide Delegate} refactoring to be correct \vsnote*{@Ole: isn't that just the default one?}{wrt. the given postconditions}.
In the following section we will detail the changes needed to accomplish this.}

%%% Local Variables:
%%% mode: latex
%%% eval: (auto-fill-mode -1)
%%% TeX-master: "../main"
%%% End:
