\oanote{find a way to cite the draft paper here p.7}
\vsnote{I don't think you can.}

Succinctly, exactly as stated by Steinfhöfel in their Ph.D. thesis, ``Abstract Execution'' denotes the idea to process abstract programs by symbolic execution (SE)~\cite{steinhoefel-20}.
SE~\cite{DBLP:journals/csur/BaldoniCDDF18,DBLP:journals/ac/YangFBCW19} abstracts concrete execution by means of symbolic representations of language runtime
state in place of concrete machine representations of such artefacts.
Thus, e.g., values, a store or a program counter all have a symbolic representation in SE.
Branching points, such as encountered when symbolically executing e.g., an if-then-else statement splits an execution path into new paths for each possible branch arm, and for each such
path e.g., the symbolic store may be preserved in the new paths, but different conditions may also be carried through such that in the path where the symbolic program counter refers
to the then branch the evaluation of the boolean expression in the if statement must be valid whereas it is not valid in the path where the symbolic program counter refers to the else branch.
Possible executions are not just captured through branching paths, but along a path itself through the symbolic store; a symbolic value represents any valid concrete substitution.

\oanote*{feel free to condense this to a single sentence}{
The SE found in KeY and REFINITY operate on a dynamic logic JavaDL, for a restricted subset of Java.
Syntactically JavaDL is an extension of first-order logic with program variables and program modalities.
Semantically JavaDL formulas are not evaluated in a single first-order structure, but a Kripke structure,
which is a collection of first-order structures \cite{DBLP:series/lncs/10001}.}

The use of SE to potentially explore every possible execution a program can have is a popular program analysis technique (cite draft paper \cite{xxx}). AE extends SE by
introducing abstract program elements (APEs) to the base language that is symbolically executed.
For statements and expressions a corresponding abstract statement and abstract expression are introduced.
APEs represent any possible substitution with concrete program elements, be it statements or expressions, from the base language being symbolically executed.
Execution of APEs is then the leap taken in AE over traditional SE and require the introduction of abstract state changes, SE branching for
any abrupt completion an APE may have (e.g. exceptions thrown), \oanote*{should at least be paraphrased}{over-approximation of returned values and thrown exceptions by symbols created ``dependently fresh'' for identifiers of abstract program placeholders} and a way to specifify the behavior of APEs ~\cite{steinhoefel-20}.

We summarize how AE is implemented in REFINITY by showing how to specify a method \rcode{void n()} where we do not want to specify the exact contents of its method body.
Only that it could possibly assign to some abstract locations (its frame) and access some abstract locations (its footprint).
Furthermore we wish to specify that no exceptions will be thrown by the method body.
We achieve such a specification in REFINITY by supplying it a definition of \rcode{n()} as seen in Listing \ref{lst:ExtractVariable-refinity-method}.
A method is specified and its only content is an abstract statement \rcode{\\abstract\_statement N;} preceeded by a comment where every line in the
comment begins with a \rcode{@}-character which denotes that this is a specification, and in this case a specification for the abstract statement.
The specification reads straigthforward with no surprises. An execution of the abstract statement \rcode{N} may have the effects of: Assigning to the
abstract location set \rcode{frN}, accessing the abstract location set \rcode{fpN} and may not throw any exceptions.

An abstract location set represents a fixed set of memory locations a program may read or write to.
The set is fixed through a program's duration but the values at these locations may change.
When an abstract location set is placed in an assignable or accessible specification it is to be understood as an upper bound;
the locations may possibly all be accessed or assigned to, not at all or anything inbetween.

\begin{figure}[h]
  \centering
  \begin{minipage}{.65\linewidth}
  \lstinputlisting[linerange={2-8},style=refinity]{ExtractVariable/REFINITY/method.refinity}
  \captionof{lstlisting}{Method}
  \label{lst:ExtractVariable-refinity-method}
  \end{minipage}
\end{figure}

%\begin{itemize}
%\item in the following, we show a simple example for AE where we prove a new refactoring correct with REFINITY
%\item with AE, we can specify a fully abstract method $n()$ (abstract body, explain locationsets, frame/footprint, global AE constraints -- do not specialise for Extract LV yet).
%\item And now we can formulate the equivalence-question in Refinity with a before/after. Point out condition under which this is not correct,
%\item the first attempt fails with open goals (give interpretation of open goal)
%\item add AE-constraint to make the example go through
%\item 1-2 sentences on ``instantiation-problem'', i.e. avoid giving specs that can't be implemented (example: assignable nothing) although technically Dom covered this already.
%\end{itemize}

In the following, we show an example of how AE can be used to prove a refactoring correct with REFINITY. 

%%% Local Variables:
%%% mode: latex
%%% eval: (auto-fill-mode -1)
%%% TeX-master: "main"
%%% End:
